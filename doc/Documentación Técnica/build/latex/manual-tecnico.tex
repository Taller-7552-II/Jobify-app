% Generated by Sphinx.
\def\sphinxdocclass{report}
\newif\ifsphinxKeepOldNames \sphinxKeepOldNamestrue
\documentclass[letterpaper,10pt,english]{sphinxmanual}
\usepackage{iftex}

\ifPDFTeX
  \usepackage[utf8]{inputenc}
\fi
\ifdefined\DeclareUnicodeCharacter
  \DeclareUnicodeCharacter{00A0}{\nobreakspace}
\fi
\usepackage{cmap}
\usepackage[T1]{fontenc}
\usepackage{amsmath,amssymb,amstext}
\usepackage{babel}
\usepackage{times}
\usepackage[Bjarne]{fncychap}
\usepackage{longtable}
\usepackage{sphinx}
\usepackage{multirow}
\usepackage{eqparbox}


\addto\captionsenglish{\renewcommand{\figurename}{Fig.\@ }}
\addto\captionsenglish{\renewcommand{\tablename}{Table }}
\SetupFloatingEnvironment{literal-block}{name=Listing }

\addto\extrasenglish{\def\pageautorefname{page}}

\setcounter{tocdepth}{1}


\title{manual-tecnico Documentation}
\date{Dec 06, 2016}
\release{1.0}
\author{Gonzalo Guzzardi, Matias Carballo, Juan Patricio Marshall}
\newcommand{\sphinxlogo}{}
\renewcommand{\releasename}{Release}
\makeindex

\makeatletter
\def\PYG@reset{\let\PYG@it=\relax \let\PYG@bf=\relax%
    \let\PYG@ul=\relax \let\PYG@tc=\relax%
    \let\PYG@bc=\relax \let\PYG@ff=\relax}
\def\PYG@tok#1{\csname PYG@tok@#1\endcsname}
\def\PYG@toks#1+{\ifx\relax#1\empty\else%
    \PYG@tok{#1}\expandafter\PYG@toks\fi}
\def\PYG@do#1{\PYG@bc{\PYG@tc{\PYG@ul{%
    \PYG@it{\PYG@bf{\PYG@ff{#1}}}}}}}
\def\PYG#1#2{\PYG@reset\PYG@toks#1+\relax+\PYG@do{#2}}

\expandafter\def\csname PYG@tok@gd\endcsname{\def\PYG@tc##1{\textcolor[rgb]{0.63,0.00,0.00}{##1}}}
\expandafter\def\csname PYG@tok@gu\endcsname{\let\PYG@bf=\textbf\def\PYG@tc##1{\textcolor[rgb]{0.50,0.00,0.50}{##1}}}
\expandafter\def\csname PYG@tok@gt\endcsname{\def\PYG@tc##1{\textcolor[rgb]{0.00,0.27,0.87}{##1}}}
\expandafter\def\csname PYG@tok@gs\endcsname{\let\PYG@bf=\textbf}
\expandafter\def\csname PYG@tok@gr\endcsname{\def\PYG@tc##1{\textcolor[rgb]{1.00,0.00,0.00}{##1}}}
\expandafter\def\csname PYG@tok@cm\endcsname{\let\PYG@it=\textit\def\PYG@tc##1{\textcolor[rgb]{0.25,0.50,0.56}{##1}}}
\expandafter\def\csname PYG@tok@vg\endcsname{\def\PYG@tc##1{\textcolor[rgb]{0.73,0.38,0.84}{##1}}}
\expandafter\def\csname PYG@tok@vi\endcsname{\def\PYG@tc##1{\textcolor[rgb]{0.73,0.38,0.84}{##1}}}
\expandafter\def\csname PYG@tok@mh\endcsname{\def\PYG@tc##1{\textcolor[rgb]{0.13,0.50,0.31}{##1}}}
\expandafter\def\csname PYG@tok@cs\endcsname{\def\PYG@tc##1{\textcolor[rgb]{0.25,0.50,0.56}{##1}}\def\PYG@bc##1{\setlength{\fboxsep}{0pt}\colorbox[rgb]{1.00,0.94,0.94}{\strut ##1}}}
\expandafter\def\csname PYG@tok@ge\endcsname{\let\PYG@it=\textit}
\expandafter\def\csname PYG@tok@vc\endcsname{\def\PYG@tc##1{\textcolor[rgb]{0.73,0.38,0.84}{##1}}}
\expandafter\def\csname PYG@tok@il\endcsname{\def\PYG@tc##1{\textcolor[rgb]{0.13,0.50,0.31}{##1}}}
\expandafter\def\csname PYG@tok@go\endcsname{\def\PYG@tc##1{\textcolor[rgb]{0.20,0.20,0.20}{##1}}}
\expandafter\def\csname PYG@tok@cp\endcsname{\def\PYG@tc##1{\textcolor[rgb]{0.00,0.44,0.13}{##1}}}
\expandafter\def\csname PYG@tok@gi\endcsname{\def\PYG@tc##1{\textcolor[rgb]{0.00,0.63,0.00}{##1}}}
\expandafter\def\csname PYG@tok@gh\endcsname{\let\PYG@bf=\textbf\def\PYG@tc##1{\textcolor[rgb]{0.00,0.00,0.50}{##1}}}
\expandafter\def\csname PYG@tok@ni\endcsname{\let\PYG@bf=\textbf\def\PYG@tc##1{\textcolor[rgb]{0.84,0.33,0.22}{##1}}}
\expandafter\def\csname PYG@tok@nl\endcsname{\let\PYG@bf=\textbf\def\PYG@tc##1{\textcolor[rgb]{0.00,0.13,0.44}{##1}}}
\expandafter\def\csname PYG@tok@nn\endcsname{\let\PYG@bf=\textbf\def\PYG@tc##1{\textcolor[rgb]{0.05,0.52,0.71}{##1}}}
\expandafter\def\csname PYG@tok@no\endcsname{\def\PYG@tc##1{\textcolor[rgb]{0.38,0.68,0.84}{##1}}}
\expandafter\def\csname PYG@tok@na\endcsname{\def\PYG@tc##1{\textcolor[rgb]{0.25,0.44,0.63}{##1}}}
\expandafter\def\csname PYG@tok@nb\endcsname{\def\PYG@tc##1{\textcolor[rgb]{0.00,0.44,0.13}{##1}}}
\expandafter\def\csname PYG@tok@nc\endcsname{\let\PYG@bf=\textbf\def\PYG@tc##1{\textcolor[rgb]{0.05,0.52,0.71}{##1}}}
\expandafter\def\csname PYG@tok@nd\endcsname{\let\PYG@bf=\textbf\def\PYG@tc##1{\textcolor[rgb]{0.33,0.33,0.33}{##1}}}
\expandafter\def\csname PYG@tok@ne\endcsname{\def\PYG@tc##1{\textcolor[rgb]{0.00,0.44,0.13}{##1}}}
\expandafter\def\csname PYG@tok@nf\endcsname{\def\PYG@tc##1{\textcolor[rgb]{0.02,0.16,0.49}{##1}}}
\expandafter\def\csname PYG@tok@si\endcsname{\let\PYG@it=\textit\def\PYG@tc##1{\textcolor[rgb]{0.44,0.63,0.82}{##1}}}
\expandafter\def\csname PYG@tok@s2\endcsname{\def\PYG@tc##1{\textcolor[rgb]{0.25,0.44,0.63}{##1}}}
\expandafter\def\csname PYG@tok@nt\endcsname{\let\PYG@bf=\textbf\def\PYG@tc##1{\textcolor[rgb]{0.02,0.16,0.45}{##1}}}
\expandafter\def\csname PYG@tok@nv\endcsname{\def\PYG@tc##1{\textcolor[rgb]{0.73,0.38,0.84}{##1}}}
\expandafter\def\csname PYG@tok@s1\endcsname{\def\PYG@tc##1{\textcolor[rgb]{0.25,0.44,0.63}{##1}}}
\expandafter\def\csname PYG@tok@ch\endcsname{\let\PYG@it=\textit\def\PYG@tc##1{\textcolor[rgb]{0.25,0.50,0.56}{##1}}}
\expandafter\def\csname PYG@tok@m\endcsname{\def\PYG@tc##1{\textcolor[rgb]{0.13,0.50,0.31}{##1}}}
\expandafter\def\csname PYG@tok@gp\endcsname{\let\PYG@bf=\textbf\def\PYG@tc##1{\textcolor[rgb]{0.78,0.36,0.04}{##1}}}
\expandafter\def\csname PYG@tok@sh\endcsname{\def\PYG@tc##1{\textcolor[rgb]{0.25,0.44,0.63}{##1}}}
\expandafter\def\csname PYG@tok@ow\endcsname{\let\PYG@bf=\textbf\def\PYG@tc##1{\textcolor[rgb]{0.00,0.44,0.13}{##1}}}
\expandafter\def\csname PYG@tok@sx\endcsname{\def\PYG@tc##1{\textcolor[rgb]{0.78,0.36,0.04}{##1}}}
\expandafter\def\csname PYG@tok@bp\endcsname{\def\PYG@tc##1{\textcolor[rgb]{0.00,0.44,0.13}{##1}}}
\expandafter\def\csname PYG@tok@c1\endcsname{\let\PYG@it=\textit\def\PYG@tc##1{\textcolor[rgb]{0.25,0.50,0.56}{##1}}}
\expandafter\def\csname PYG@tok@o\endcsname{\def\PYG@tc##1{\textcolor[rgb]{0.40,0.40,0.40}{##1}}}
\expandafter\def\csname PYG@tok@kc\endcsname{\let\PYG@bf=\textbf\def\PYG@tc##1{\textcolor[rgb]{0.00,0.44,0.13}{##1}}}
\expandafter\def\csname PYG@tok@c\endcsname{\let\PYG@it=\textit\def\PYG@tc##1{\textcolor[rgb]{0.25,0.50,0.56}{##1}}}
\expandafter\def\csname PYG@tok@mf\endcsname{\def\PYG@tc##1{\textcolor[rgb]{0.13,0.50,0.31}{##1}}}
\expandafter\def\csname PYG@tok@err\endcsname{\def\PYG@bc##1{\setlength{\fboxsep}{0pt}\fcolorbox[rgb]{1.00,0.00,0.00}{1,1,1}{\strut ##1}}}
\expandafter\def\csname PYG@tok@mb\endcsname{\def\PYG@tc##1{\textcolor[rgb]{0.13,0.50,0.31}{##1}}}
\expandafter\def\csname PYG@tok@ss\endcsname{\def\PYG@tc##1{\textcolor[rgb]{0.32,0.47,0.09}{##1}}}
\expandafter\def\csname PYG@tok@sr\endcsname{\def\PYG@tc##1{\textcolor[rgb]{0.14,0.33,0.53}{##1}}}
\expandafter\def\csname PYG@tok@mo\endcsname{\def\PYG@tc##1{\textcolor[rgb]{0.13,0.50,0.31}{##1}}}
\expandafter\def\csname PYG@tok@kd\endcsname{\let\PYG@bf=\textbf\def\PYG@tc##1{\textcolor[rgb]{0.00,0.44,0.13}{##1}}}
\expandafter\def\csname PYG@tok@mi\endcsname{\def\PYG@tc##1{\textcolor[rgb]{0.13,0.50,0.31}{##1}}}
\expandafter\def\csname PYG@tok@kn\endcsname{\let\PYG@bf=\textbf\def\PYG@tc##1{\textcolor[rgb]{0.00,0.44,0.13}{##1}}}
\expandafter\def\csname PYG@tok@cpf\endcsname{\let\PYG@it=\textit\def\PYG@tc##1{\textcolor[rgb]{0.25,0.50,0.56}{##1}}}
\expandafter\def\csname PYG@tok@kr\endcsname{\let\PYG@bf=\textbf\def\PYG@tc##1{\textcolor[rgb]{0.00,0.44,0.13}{##1}}}
\expandafter\def\csname PYG@tok@s\endcsname{\def\PYG@tc##1{\textcolor[rgb]{0.25,0.44,0.63}{##1}}}
\expandafter\def\csname PYG@tok@kp\endcsname{\def\PYG@tc##1{\textcolor[rgb]{0.00,0.44,0.13}{##1}}}
\expandafter\def\csname PYG@tok@w\endcsname{\def\PYG@tc##1{\textcolor[rgb]{0.73,0.73,0.73}{##1}}}
\expandafter\def\csname PYG@tok@kt\endcsname{\def\PYG@tc##1{\textcolor[rgb]{0.56,0.13,0.00}{##1}}}
\expandafter\def\csname PYG@tok@sc\endcsname{\def\PYG@tc##1{\textcolor[rgb]{0.25,0.44,0.63}{##1}}}
\expandafter\def\csname PYG@tok@sb\endcsname{\def\PYG@tc##1{\textcolor[rgb]{0.25,0.44,0.63}{##1}}}
\expandafter\def\csname PYG@tok@k\endcsname{\let\PYG@bf=\textbf\def\PYG@tc##1{\textcolor[rgb]{0.00,0.44,0.13}{##1}}}
\expandafter\def\csname PYG@tok@se\endcsname{\let\PYG@bf=\textbf\def\PYG@tc##1{\textcolor[rgb]{0.25,0.44,0.63}{##1}}}
\expandafter\def\csname PYG@tok@sd\endcsname{\let\PYG@it=\textit\def\PYG@tc##1{\textcolor[rgb]{0.25,0.44,0.63}{##1}}}

\def\PYGZbs{\char`\\}
\def\PYGZus{\char`\_}
\def\PYGZob{\char`\{}
\def\PYGZcb{\char`\}}
\def\PYGZca{\char`\^}
\def\PYGZam{\char`\&}
\def\PYGZlt{\char`\<}
\def\PYGZgt{\char`\>}
\def\PYGZsh{\char`\#}
\def\PYGZpc{\char`\%}
\def\PYGZdl{\char`\$}
\def\PYGZhy{\char`\-}
\def\PYGZsq{\char`\'}
\def\PYGZdq{\char`\"}
\def\PYGZti{\char`\~}
% for compatibility with earlier versions
\def\PYGZat{@}
\def\PYGZlb{[}
\def\PYGZrb{]}
\makeatother

\renewcommand\PYGZsq{\textquotesingle}

\begin{document}

\maketitle
\tableofcontents
\phantomsection\label{index::doc}


Contents:


\chapter{Activity}
\label{Activities/package-index:welcome-to-jobify-s-technical-manual-documentation}\label{Activities/package-index::doc}\label{Activities/package-index:activity}

\section{MainScreenActivity}
\label{Activities/MainScreenActivity:mainscreenactivity}\label{Activities/MainScreenActivity::doc}\index{MainScreenActivity (Java class)}

\begin{fulllineitems}
\phantomsection\label{Activities/MainScreenActivity:com.fiuba.tallerii.jobify.MainScreenActivity}\pysigline{public class \sphinxbfcode{MainScreenActivity} extends AppCompatActivity}
Actividad principal de la aplicación. Contiene 4 Fragments correspondientes a las principales ventanas de la aplicación: Perfil, Contactos, Notificaciones, Chats. Para manipular dichos Fragments, utiliza la clase \sphinxtitleref{ViewPagerAdapter}.

\end{fulllineitems}



\subsection{Fields}
\label{Activities/MainScreenActivity:fields}

\subsubsection{mToolbar}
\label{Activities/MainScreenActivity:mtoolbar}\index{mToolbar (Java field)}

\begin{fulllineitems}
\phantomsection\label{Activities/MainScreenActivity:com.fiuba.tallerii.jobify.MainScreenActivity.mToolbar}\pysigline{private Toolbar \sphinxbfcode{mToolbar}}
\end{fulllineitems}



\subsubsection{mTabLayout}
\label{Activities/MainScreenActivity:mtablayout}\index{mTabLayout (Java field)}

\begin{fulllineitems}
\phantomsection\label{Activities/MainScreenActivity:com.fiuba.tallerii.jobify.MainScreenActivity.mTabLayout}\pysigline{private TabLayout \sphinxbfcode{mTabLayout}}
Referencia al layout que contiene y establece como se muestran los tabs.

\end{fulllineitems}



\subsubsection{mViewPager}
\label{Activities/MainScreenActivity:mviewpager}\index{mViewPager (Java field)}

\begin{fulllineitems}
\phantomsection\label{Activities/MainScreenActivity:com.fiuba.tallerii.jobify.MainScreenActivity.mViewPager}\pysigline{private ViewPager \sphinxbfcode{mViewPager}}
ViewPager que maneja el paginado de los distintos Fragments que componen a la Activity.

\end{fulllineitems}



\subsection{Methods}
\label{Activities/MainScreenActivity:methods}

\subsubsection{setupTabIcons}
\label{Activities/MainScreenActivity:setuptabicons}\index{setupTabIcons() (Java method)}

\begin{fulllineitems}
\phantomsection\label{Activities/MainScreenActivity:com.fiuba.tallerii.jobify.MainScreenActivity.setupTabIcons()}\pysiglinewithargsret{private void \sphinxbfcode{setupTabIcons}}{}{}
Establece las asociaciones entre íconos y Tabs.

\end{fulllineitems}



\subsubsection{setupViewPager}
\label{Activities/MainScreenActivity:setupviewpager}\index{setupViewPager(ViewPager) (Java method)}

\begin{fulllineitems}
\phantomsection\label{Activities/MainScreenActivity:com.fiuba.tallerii.jobify.MainScreenActivity.setupViewPager(ViewPager)}\pysiglinewithargsret{private void \sphinxbfcode{setupViewPager}}{ViewPager\emph{ viewPager}}{}
Crea todos los Fragments a utilizar en la MainScreenActivity y los agrega a un \sphinxtitleref{ViewPagerAdapter}, para luego asignarselo al viewPager pasado como parámetro.
\begin{quote}\begin{description}
\item[{Parameters}] \leavevmode\begin{itemize}
\item {} 
\textbf{\texttt{viewPager}} -- Referencia al viewPager utiliado para manejar el paginado de los Fragments. A esta ViewPager se le seteará el Adapter \sphinxtitleref{ViewPagerAdapter} específico de la aplicación.

\end{itemize}

\end{description}\end{quote}

\end{fulllineitems}



\section{SingleFragmentActivity}
\label{Activities/SingleFragmentActivity::doc}\label{Activities/SingleFragmentActivity:singlefragmentactivity}\index{SingleFragmentActivity (Java class)}

\begin{fulllineitems}
\phantomsection\label{Activities/SingleFragmentActivity:com.fiuba.tallerii.jobify.SingleFragmentActivity}\pysigline{public class \sphinxbfcode{SingleFragmentActivity} extends AppCompatActivity}
Actividad inicial que contiene un único Fragment para manejar su contenido visible. SingleFragmentActivity es una clase abstract

\end{fulllineitems}



\subsection{Methods}
\label{Activities/SingleFragmentActivity:methods}

\subsubsection{createFragment}
\label{Activities/SingleFragmentActivity:createfragment}\index{createFragment() (Java method)}

\begin{fulllineitems}
\phantomsection\label{Activities/SingleFragmentActivity:com.fiuba.tallerii.jobify.SingleFragmentActivity.createFragment()}\pysiglinewithargsret{protected abstract Fragment \sphinxbfcode{createFragment}}{}{}
Metodo que deben sobreescribir todas las clases que heredan de SingleFragmentActivity, para definir el Fragment que se utilizará.
\begin{quote}\begin{description}
\item[{Returns}] \leavevmode
nuevo Fragment que se utilizará para manejar la interfaz de usuario de la activity.

\end{description}\end{quote}

\end{fulllineitems}



\subsubsection{onCreate}
\label{Activities/SingleFragmentActivity:oncreate}\index{onCreate(Bundle) (Java method)}

\begin{fulllineitems}
\phantomsection\label{Activities/SingleFragmentActivity:com.fiuba.tallerii.jobify.SingleFragmentActivity.onCreate(Bundle)}\pysiglinewithargsret{protected void \sphinxbfcode{onCreate}}{Bundle\emph{ savedInstanceState}}{}
Inicializa la activity y crea un Fragment mediante el método abstracto createFragment(). Luego configura dicho Fragment para ser utilizado como la interfaz de usuario de esta Activity.
\begin{quote}\begin{description}
\item[{Parameters}] \leavevmode\begin{itemize}
\item {} 
\textbf{\texttt{savedInstanceState}} -- 

\end{itemize}

\end{description}\end{quote}

\end{fulllineitems}



\section{LogInActivity}
\label{Activities/LogInActivity:loginactivity}\label{Activities/LogInActivity::doc}\index{LogInActivity (Java class)}

\begin{fulllineitems}
\phantomsection\label{Activities/LogInActivity:com.fiuba.tallerii.jobify.LogInActivity}\pysigline{public class \sphinxbfcode{LogInActivity} extends {\hyperref[Activities/SingleFragmentActivity:com.fiuba.tallerii.jobify.SingleFragmentActivity]{\sphinxcrossref{SingleFragmentActivity}}}}
Actividad en la que inicia la aplicación, en la cual el usuario puede loguearse para ingresar a la aplicación, o crearse una nueva cuenta. Crearse una nueva cuenta iniciará una nueva \sphinxtitleref{SignUpActivity}.

\end{fulllineitems}



\subsection{Methods}
\label{Activities/LogInActivity:methods}

\subsubsection{createFragment}
\label{Activities/LogInActivity:createfragment}\index{createFragment() (Java method)}

\begin{fulllineitems}
\phantomsection\label{Activities/LogInActivity:com.fiuba.tallerii.jobify.LogInActivity.createFragment()}\pysiglinewithargsret{protected Fragment \sphinxbfcode{createFragment}}{}{}
Crea el Fragment que se utilizará en la Activity.
\begin{quote}\begin{description}
\item[{Returns}] \leavevmode
\sphinxtitleref{LogInFragment} en forma de Fragment

\end{description}\end{quote}

\end{fulllineitems}



\section{SignUpActivity}
\label{Activities/SignUpActivity::doc}\label{Activities/SignUpActivity:signupactivity}\index{SignUpActivity (Java class)}

\begin{fulllineitems}
\phantomsection\label{Activities/SignUpActivity:com.fiuba.tallerii.jobify.SignUpActivity}\pysigline{public class \sphinxbfcode{SignUpActivity} extends {\hyperref[Activities/SingleFragmentActivity:com.fiuba.tallerii.jobify.SingleFragmentActivity]{\sphinxcrossref{SingleFragmentActivity}}}}
Actividad con la interfaz para crear una nueva cuenta en la aplicación

\end{fulllineitems}



\subsection{Methods}
\label{Activities/SignUpActivity:methods}

\subsubsection{createFragment}
\label{Activities/SignUpActivity:createfragment}\index{createFragment() (Java method)}

\begin{fulllineitems}
\phantomsection\label{Activities/SignUpActivity:com.fiuba.tallerii.jobify.SignUpActivity.createFragment()}\pysiglinewithargsret{protected Fragment \sphinxbfcode{createFragment}}{}{}
Crea el Fragment que se utilizará en la Activity.
\begin{quote}\begin{description}
\item[{Returns}] \leavevmode
\sphinxtitleref{SignUpFragment} en forma de Fragment

\end{description}\end{quote}

\end{fulllineitems}



\section{AddContactActivity}
\label{Activities/AddContactActivity::doc}\label{Activities/AddContactActivity:addcontactactivity}\index{AddContactActivity (Java class)}

\begin{fulllineitems}
\phantomsection\label{Activities/AddContactActivity:com.fiuba.tallerii.jobify.AddContactActivity}\pysigline{public class \sphinxbfcode{AddContactActivity} extends {\hyperref[Activities/SingleFragmentActivity:com.fiuba.tallerii.jobify.SingleFragmentActivity]{\sphinxcrossref{SingleFragmentActivity}}}}
Actividad con la interfaz para agregar un nuevo contacto

\end{fulllineitems}



\subsection{Methods}
\label{Activities/AddContactActivity:methods}

\subsubsection{createFragment}
\label{Activities/AddContactActivity:createfragment}\index{createFragment() (Java method)}

\begin{fulllineitems}
\phantomsection\label{Activities/AddContactActivity:com.fiuba.tallerii.jobify.AddContactActivity.createFragment()}\pysiglinewithargsret{protected Fragment \sphinxbfcode{createFragment}}{}{}
Crea el Fragment que se utilizará en la Activity.
\begin{quote}\begin{description}
\item[{Returns}] \leavevmode
\sphinxtitleref{AddContactFragment} en forma de Fragment

\end{description}\end{quote}

\end{fulllineitems}



\section{SkillsActivity}
\label{Activities/SkillsActivity:skillsactivity}\label{Activities/SkillsActivity::doc}\index{SkillsActivity (Java class)}

\begin{fulllineitems}
\phantomsection\label{Activities/SkillsActivity:com.fiuba.tallerii.jobify.SkillsActivity}\pysigline{public class \sphinxbfcode{SkillsActivity} extends {\hyperref[Activities/SingleFragmentActivity:com.fiuba.tallerii.jobify.SingleFragmentActivity]{\sphinxcrossref{SingleFragmentActivity}}}}
Actividad con la interfaz para mostrar y editar las destrezas

\end{fulllineitems}



\subsection{Methods}
\label{Activities/SkillsActivity:methods}

\subsubsection{createFragment}
\label{Activities/SkillsActivity:createfragment}\index{createFragment() (Java method)}

\begin{fulllineitems}
\phantomsection\label{Activities/SkillsActivity:com.fiuba.tallerii.jobify.SkillsActivity.createFragment()}\pysiglinewithargsret{protected Fragment \sphinxbfcode{createFragment}}{}{}
Crea el Fragment que se utilizará en la Activity.
\begin{quote}\begin{description}
\item[{Returns}] \leavevmode
\sphinxtitleref{SkillsFragment} en forma de Fragment

\end{description}\end{quote}

\end{fulllineitems}



\section{AddSkillActivity}
\label{Activities/AddSkillActivity::doc}\label{Activities/AddSkillActivity:addskillactivity}\index{AddSkillActivity (Java class)}

\begin{fulllineitems}
\phantomsection\label{Activities/AddSkillActivity:com.fiuba.tallerii.jobify.AddSkillActivity}\pysigline{public class \sphinxbfcode{AddSkillActivity} extends {\hyperref[Activities/SingleFragmentActivity:com.fiuba.tallerii.jobify.SingleFragmentActivity]{\sphinxcrossref{SingleFragmentActivity}}}}
Actividad con la interfaz para agregar una nueva destreza

\end{fulllineitems}



\subsection{Methods}
\label{Activities/AddSkillActivity:methods}

\subsubsection{createFragment}
\label{Activities/AddSkillActivity:createfragment}\index{createFragment() (Java method)}

\begin{fulllineitems}
\phantomsection\label{Activities/AddSkillActivity:com.fiuba.tallerii.jobify.AddSkillActivity.createFragment()}\pysiglinewithargsret{protected Fragment \sphinxbfcode{createFragment}}{}{}
Crea el Fragment que se utilizará en la Activity.
\begin{quote}\begin{description}
\item[{Returns}] \leavevmode
\sphinxtitleref{AddSkillFragment} en forma de Fragment

\end{description}\end{quote}

\end{fulllineitems}



\section{JobsActivity}
\label{Activities/JobsActivity:jobsactivity}\label{Activities/JobsActivity::doc}\index{JobsActivity (Java class)}

\begin{fulllineitems}
\phantomsection\label{Activities/JobsActivity:com.fiuba.tallerii.jobify.JobsActivity}\pysigline{public class \sphinxbfcode{JobsActivity} extends {\hyperref[Activities/SingleFragmentActivity:com.fiuba.tallerii.jobify.SingleFragmentActivity]{\sphinxcrossref{SingleFragmentActivity}}}}
Actividad con la interfaz para mostrar y editar las experiencias laborales

\end{fulllineitems}



\subsection{Methods}
\label{Activities/JobsActivity:methods}

\subsubsection{createFragment}
\label{Activities/JobsActivity:createfragment}\index{createFragment() (Java method)}

\begin{fulllineitems}
\phantomsection\label{Activities/JobsActivity:com.fiuba.tallerii.jobify.JobsActivity.createFragment()}\pysiglinewithargsret{protected Fragment \sphinxbfcode{createFragment}}{}{}
Crea el Fragment que se utilizará en la Activity.
\begin{quote}\begin{description}
\item[{Returns}] \leavevmode
\sphinxtitleref{JobsFragment} en forma de Fragment

\end{description}\end{quote}

\end{fulllineitems}



\section{AddJobActivity}
\label{Activities/AddJobActivity:addjobactivity}\label{Activities/AddJobActivity::doc}\index{AddJobActivity (Java class)}

\begin{fulllineitems}
\phantomsection\label{Activities/AddJobActivity:AddJobActivity}\pysigline{public class \sphinxbfcode{AddJobActivity} extends {\hyperref[Activities/SingleFragmentActivity:com.fiuba.tallerii.jobify.SingleFragmentActivity]{\sphinxcrossref{SingleFragmentActivity}}}}
Actividad con la interfaz para agregar una nueva experiencia laboral

\end{fulllineitems}



\subsection{Methods}
\label{Activities/AddJobActivity:methods}

\subsubsection{createFragment}
\label{Activities/AddJobActivity:createfragment}\index{createFragment() (Java method)}

\begin{fulllineitems}
\phantomsection\label{Activities/AddJobActivity:AddJobActivity.createFragment()}\pysiglinewithargsret{protected Fragment \sphinxbfcode{createFragment}}{}{}
Crea el Fragment que se utilizará en la Activity.
\begin{quote}\begin{description}
\item[{Returns}] \leavevmode
\sphinxtitleref{AddJobFragment} en forma de Fragment

\end{description}\end{quote}

\end{fulllineitems}



\chapter{Fragments}
\label{Fragments/package-index:fragments}\label{Fragments/package-index::doc}

\section{LogInFragment}
\label{Fragments/LogInFragment:loginfragment}\label{Fragments/LogInFragment::doc}\index{LogInFragment (Java class)}

\begin{fulllineitems}
\phantomsection\label{Fragments/LogInFragment:com.fiuba.tallerii.jobify.LogInFragment}\pysigline{public class \sphinxbfcode{LogInFragment} extends Fragment implements LoaderManager.LoaderCallbacks\textless{}Cursor\textgreater{}}
Establece la interfaz de usuario para loguearse a la aplicación o crearse cuenta de no poseer una.

\end{fulllineitems}



\subsection{Fields}
\label{Fragments/LogInFragment:fields}

\subsubsection{mAuthTask}
\label{Fragments/LogInFragment:mauthtask}\index{mAuthTask (Java field)}

\begin{fulllineitems}
\phantomsection\label{Fragments/LogInFragment:com.fiuba.tallerii.jobify.LogInFragment.mAuthTask}\pysigline{private UserLoginTask \sphinxbfcode{mAuthTask}}
Task asincrónica para intentar loguearse al servidor sin detener la UI thread.

\end{fulllineitems}



\subsubsection{mEmailAutocompleteText}
\label{Fragments/LogInFragment:memailautocompletetext}\index{mEmailAutocompleteText (Java field)}

\begin{fulllineitems}
\phantomsection\label{Fragments/LogInFragment:com.fiuba.tallerii.jobify.LogInFragment.mEmailAutocompleteText}\pysigline{private AutoCompleteTextView \sphinxbfcode{mEmailAutocompleteText}}
Referencia al campo EditText que contiene la forma para ingresar el email.

\end{fulllineitems}



\subsubsection{mPasswordEditText}
\label{Fragments/LogInFragment:mpasswordedittext}\index{mPasswordEditText (Java field)}

\begin{fulllineitems}
\phantomsection\label{Fragments/LogInFragment:com.fiuba.tallerii.jobify.LogInFragment.mPasswordEditText}\pysigline{private EditText \sphinxbfcode{mPasswordEditText}}
Referencia al campo EditText que contiene la forma para ingresar el password

\end{fulllineitems}



\subsubsection{mSignInButton}
\label{Fragments/LogInFragment:msigninbutton}\index{mSignInButton (Java field)}

\begin{fulllineitems}
\phantomsection\label{Fragments/LogInFragment:com.fiuba.tallerii.jobify.LogInFragment.mSignInButton}\pysigline{private Button \sphinxbfcode{mSignInButton}}
Referencia al botón de Sign Up utilizado para iniciar una \sphinxtitleref{SignUpActivity}.

\end{fulllineitems}



\subsubsection{mSignUpButton}
\label{Fragments/LogInFragment:msignupbutton}\index{mSignUpButton (Java field)}

\begin{fulllineitems}
\phantomsection\label{Fragments/LogInFragment:com.fiuba.tallerii.jobify.LogInFragment.mSignUpButton}\pysigline{private Button \sphinxbfcode{mSignUpButton}}
Referencia al botón de Log In utilizado para intentar acceder a la aplicación.

\end{fulllineitems}



\subsection{Methods}
\label{Fragments/LogInFragment:methods}

\subsubsection{onCreateView}
\label{Fragments/LogInFragment:oncreateview}

\subsubsection{joinApplication}
\label{Fragments/LogInFragment:joinapplication}\index{joinApplication() (Java method)}

\begin{fulllineitems}
\phantomsection\label{Fragments/LogInFragment:com.fiuba.tallerii.jobify.LogInFragment.joinApplication()}\pysiglinewithargsret{private void \sphinxbfcode{joinApplication}}{}{}
Inicia la Activity: \sphinxtitleref{MainScreenActivity}.

\end{fulllineitems}



\subsubsection{onResume}
\label{Fragments/LogInFragment:onresume}\index{onResume() (Java method)}

\begin{fulllineitems}
\phantomsection\label{Fragments/LogInFragment:com.fiuba.tallerii.jobify.LogInFragment.onResume()}\pysiglinewithargsret{public void \sphinxbfcode{onResume}}{}{}
Invoca a \sphinxtitleref{resetFields} para limpiar los campos de los formularios.

\end{fulllineitems}



\subsubsection{resetFields}
\label{Fragments/LogInFragment:resetfields}\index{resetFields() (Java method)}

\begin{fulllineitems}
\phantomsection\label{Fragments/LogInFragment:com.fiuba.tallerii.jobify.LogInFragment.resetFields()}\pysiglinewithargsret{private void \sphinxbfcode{resetFields}}{}{}
Limpia los campos de los formularios.

\end{fulllineitems}



\subsubsection{OpenSignUpActivity}
\label{Fragments/LogInFragment:opensignupactivity}\index{OpenSignUpActivity() (Java method)}

\begin{fulllineitems}
\phantomsection\label{Fragments/LogInFragment:com.fiuba.tallerii.jobify.LogInFragment.OpenSignUpActivity()}\pysiglinewithargsret{private void \sphinxbfcode{OpenSignUpActivity}}{}{}
Inicia la Activity: \sphinxtitleref{SignUpActivity}.

\end{fulllineitems}



\subsubsection{populateAutoComplete}
\label{Fragments/LogInFragment:populateautocomplete}\index{populateAutoComplete() (Java method)}

\begin{fulllineitems}
\phantomsection\label{Fragments/LogInFragment:com.fiuba.tallerii.jobify.LogInFragment.populateAutoComplete()}\pysiglinewithargsret{private void \sphinxbfcode{populateAutoComplete}}{}{}
Sugiere emails conocidos para ingresar en el campo \sphinxtitleref{mEmailAutocompleteText}.

\end{fulllineitems}



\subsubsection{attemptLogin}
\label{Fragments/LogInFragment:attemptlogin}\index{attemptLogin() (Java method)}

\begin{fulllineitems}
\phantomsection\label{Fragments/LogInFragment:com.fiuba.tallerii.jobify.LogInFragment.attemptLogin()}\pysiglinewithargsret{private void \sphinxbfcode{attemptLogin}}{}{}
Se fija que todos los campos ingresados tengan información válida. De ser así, inicia una tarea asincrónica para intentar conectarse al servidor con el email y password ingresados.

\end{fulllineitems}



\subsubsection{showProgress}
\label{Fragments/LogInFragment:showprogress}\index{showProgress(boolean) (Java method)}

\begin{fulllineitems}
\phantomsection\label{Fragments/LogInFragment:com.fiuba.tallerii.jobify.LogInFragment.showProgress(boolean)}\pysiglinewithargsret{private void \sphinxbfcode{showProgress}}{boolean\emph{ show}}{}
Si el parámetro ingresado es true, esconde los elementos de UI de logueo y muestra la animación de cargado. Si es true, esconde la animación de cargado y regresa la aplicación a su estado ordinario.
\begin{quote}\begin{description}
\item[{Parameters}] \leavevmode\begin{itemize}
\item {} 
\textbf{\texttt{show}} -- Determina si se muestra la animación de cargando, o si se esconde.

\end{itemize}

\end{description}\end{quote}

\end{fulllineitems}



\section{SignUpFragment}
\label{Fragments/SignUpFragment:signupfragment}\label{Fragments/SignUpFragment::doc}\index{SignUpFragment (Java class)}

\begin{fulllineitems}
\phantomsection\label{Fragments/SignUpFragment:com.fiuba.tallerii.jobify.SignUpFragment}\pysigline{public class \sphinxbfcode{SignUpFragment} extends Fragment implements LoaderManager.LoaderCallbacks\textless{}Cursor\textgreater{}}
Establece la interfaz de usuario para crearse una nueva cuenta en la aplicación.

\end{fulllineitems}



\subsection{Fields}
\label{Fragments/SignUpFragment:fields}

\subsubsection{mSignUpTask}
\label{Fragments/SignUpFragment:msignuptask}\index{mSignUpTask (Java field)}

\begin{fulllineitems}
\phantomsection\label{Fragments/SignUpFragment:com.fiuba.tallerii.jobify.SignUpFragment.mSignUpTask}\pysigline{private UserSignUpTask \sphinxbfcode{mSignUpTask}}
Task asincrónica para conectarse al servidor y crearse una nueva cuenta sin detener la UI thread.

\end{fulllineitems}



\subsubsection{mEmailAutocompleteText}
\label{Fragments/SignUpFragment:memailautocompletetext}\index{mEmailAutocompleteText (Java field)}

\begin{fulllineitems}
\phantomsection\label{Fragments/SignUpFragment:com.fiuba.tallerii.jobify.SignUpFragment.mEmailAutocompleteText}\pysigline{private AutoCompleteTextView \sphinxbfcode{mEmailAutocompleteText}}
Referencia al campo EditText que contiene la forma para ingresar el email.

\end{fulllineitems}



\subsubsection{mPasswordEditText}
\label{Fragments/SignUpFragment:mpasswordedittext}\index{mPasswordEditText (Java field)}

\begin{fulllineitems}
\phantomsection\label{Fragments/SignUpFragment:com.fiuba.tallerii.jobify.SignUpFragment.mPasswordEditText}\pysigline{private EditText \sphinxbfcode{mPasswordEditText}}
Referencia al campo EditText que contiene la forma para ingresar el password

\end{fulllineitems}



\subsubsection{mFirstNameEditText}
\label{Fragments/SignUpFragment:mfirstnameedittext}\index{mFirstNameEditText (Java field)}

\begin{fulllineitems}
\phantomsection\label{Fragments/SignUpFragment:com.fiuba.tallerii.jobify.SignUpFragment.mFirstNameEditText}\pysigline{private EditText \sphinxbfcode{mFirstNameEditText}}
Referencia al campo EditText que contiene la forma para ingresar el nombre del usuario.

\end{fulllineitems}



\subsubsection{mLastNameEditText}
\label{Fragments/SignUpFragment:mlastnameedittext}\index{mLastNameEditText (Java field)}

\begin{fulllineitems}
\phantomsection\label{Fragments/SignUpFragment:com.fiuba.tallerii.jobify.SignUpFragment.mLastNameEditText}\pysigline{private EditText \sphinxbfcode{mLastNameEditText}}
Referencia al campo EditText que contiene la forma para ingresar el apellido del usuario.

\end{fulllineitems}



\subsubsection{mSignUpButton}
\label{Fragments/SignUpFragment:msignupbutton}\index{mSignUpButton (Java field)}

\begin{fulllineitems}
\phantomsection\label{Fragments/SignUpFragment:com.fiuba.tallerii.jobify.SignUpFragment.mSignUpButton}\pysigline{private Button \sphinxbfcode{mSignUpButton}}
Referencia al botón de Sign up, utilizado para confirmar los campos ingresados y crearse una nueva cuenta.

\end{fulllineitems}



\subsection{Methods}
\label{Fragments/SignUpFragment:methods}

\subsubsection{onCreateView}
\label{Fragments/SignUpFragment:oncreateview}

\subsubsection{attemptSignUp}
\label{Fragments/SignUpFragment:attemptsignup}\index{attemptSignUp() (Java method)}

\begin{fulllineitems}
\phantomsection\label{Fragments/SignUpFragment:com.fiuba.tallerii.jobify.SignUpFragment.attemptSignUp()}\pysiglinewithargsret{private void \sphinxbfcode{attemptSignUp}}{}{}
Se fija que todos los campos ingresados tengan información válida. De ser así, inicia una tarea asincrónica para intentar conectarse al servidor y crear la cuenta. Si la cuenta se crea con éxito, la aplicación regresa a la Activity: \sphinxtitleref{LogInActivity}.

\end{fulllineitems}



\subsubsection{onResume}
\label{Fragments/SignUpFragment:onresume}\index{onResume() (Java method)}

\begin{fulllineitems}
\phantomsection\label{Fragments/SignUpFragment:com.fiuba.tallerii.jobify.SignUpFragment.onResume()}\pysiglinewithargsret{public void \sphinxbfcode{onResume}}{}{}
Invoca a \sphinxtitleref{resetFields} para limpiar los campos de los formularios.

\end{fulllineitems}



\subsubsection{resetFields}
\label{Fragments/SignUpFragment:resetfields}\index{resetFields() (Java method)}

\begin{fulllineitems}
\phantomsection\label{Fragments/SignUpFragment:com.fiuba.tallerii.jobify.SignUpFragment.resetFields()}\pysiglinewithargsret{private void \sphinxbfcode{resetFields}}{}{}
Limpia los campos de los formularios.

\end{fulllineitems}



\subsubsection{OpenSignUpActivity}
\label{Fragments/SignUpFragment:opensignupactivity}\index{OpenSignUpActivity() (Java method)}

\begin{fulllineitems}
\phantomsection\label{Fragments/SignUpFragment:com.fiuba.tallerii.jobify.SignUpFragment.OpenSignUpActivity()}\pysiglinewithargsret{private void \sphinxbfcode{OpenSignUpActivity}}{}{}
Inicia la Activity: \sphinxtitleref{SignUpActivity}.

\end{fulllineitems}



\subsubsection{populateAutoComplete}
\label{Fragments/SignUpFragment:populateautocomplete}\index{populateAutoComplete() (Java method)}

\begin{fulllineitems}
\phantomsection\label{Fragments/SignUpFragment:com.fiuba.tallerii.jobify.SignUpFragment.populateAutoComplete()}\pysiglinewithargsret{private void \sphinxbfcode{populateAutoComplete}}{}{}
Sugiere emails conocidos para ingresar en el campo \sphinxtitleref{mEmailAutocompleteText}.

\end{fulllineitems}



\subsubsection{showProgress}
\label{Fragments/SignUpFragment:showprogress}\index{showProgress(boolean) (Java method)}

\begin{fulllineitems}
\phantomsection\label{Fragments/SignUpFragment:com.fiuba.tallerii.jobify.SignUpFragment.showProgress(boolean)}\pysiglinewithargsret{private void \sphinxbfcode{showProgress}}{boolean\emph{ show}}{}
Si el parámetro ingresado es true, esconde los elementos de UI de logueo y muestra la animación de cargado. Si es true, esconde la animación de cargado y regresa la aplicación a su estado ordinario.
\begin{quote}\begin{description}
\item[{Parameters}] \leavevmode\begin{itemize}
\item {} 
\textbf{\texttt{show}} -- Determina si se muestra la animación de cargando, o si se esconde.

\end{itemize}

\end{description}\end{quote}

\end{fulllineitems}



\section{ProfileFragment}
\label{Fragments/ProfileFragment:profilefragment}\label{Fragments/ProfileFragment::doc}\index{ProfileFragment (Java class)}

\begin{fulllineitems}
\phantomsection\label{Fragments/ProfileFragment:com.fiuba.tallerii.jobify.ProfileFragment}\pysigline{public class \sphinxbfcode{ProfileFragment} extends Fragment}
Provee la interfaz para visualizar el perfil del usuario.

\end{fulllineitems}



\subsection{Methods}
\label{Fragments/ProfileFragment:methods}

\subsubsection{onCreateView}
\label{Fragments/ProfileFragment:oncreateview}

\subsubsection{attemptSignUp}
\label{Fragments/ProfileFragment:attemptsignup}\index{attemptSignUp() (Java method)}

\begin{fulllineitems}
\phantomsection\label{Fragments/ProfileFragment:com.fiuba.tallerii.jobify.ProfileFragment.attemptSignUp()}\pysiglinewithargsret{private void \sphinxbfcode{attemptSignUp}}{}{}
Se fija que todos los campos ingresados tengan información válida. De ser así, inicia una tarea asincrónica para intentar conectarse al servidor y crear la cuenta. Si la cuenta se crea con éxito, la aplicación regresa a la Activity: \sphinxtitleref{LogInActivity}.

\end{fulllineitems}



\subsubsection{onResume}
\label{Fragments/ProfileFragment:onresume}\index{onResume() (Java method)}

\begin{fulllineitems}
\phantomsection\label{Fragments/ProfileFragment:com.fiuba.tallerii.jobify.ProfileFragment.onResume()}\pysiglinewithargsret{public void \sphinxbfcode{onResume}}{}{}
Actualiza los campos del perfil a través del método \sphinxtitleref{updateFields}.

\end{fulllineitems}



\subsubsection{updateFields}
\label{Fragments/ProfileFragment:updatefields}\index{updateFields() (Java method)}

\begin{fulllineitems}
\phantomsection\label{Fragments/ProfileFragment:com.fiuba.tallerii.jobify.ProfileFragment.updateFields()}\pysiglinewithargsret{private void \sphinxbfcode{updateFields}}{}{}
Actualiza los valores de los componentes de la vista, utilizando los datos contenidos en el Singleto \sphinxtitleref{InformationHolder}.

\end{fulllineitems}



\subsubsection{chooseProfilePicture}
\label{Fragments/ProfileFragment:chooseprofilepicture}\index{chooseProfilePicture() (Java method)}

\begin{fulllineitems}
\phantomsection\label{Fragments/ProfileFragment:com.fiuba.tallerii.jobify.ProfileFragment.chooseProfilePicture()}\pysiglinewithargsret{private void \sphinxbfcode{chooseProfilePicture}}{}{}
Inicia un Intent para seleccionar una foto del dispositivo con cualquier aplicación capaz.

\end{fulllineitems}



\subsubsection{onActivityResult}
\label{Fragments/ProfileFragment:onactivityresult}\index{onActivityResult(int, int, Intent) (Java method)}

\begin{fulllineitems}
\phantomsection\label{Fragments/ProfileFragment:com.fiuba.tallerii.jobify.ProfileFragment.onActivityResult(int, int, Intent)}\pysiglinewithargsret{public void \sphinxbfcode{onActivityResult}}{int\emph{ requestCode}, int\emph{ resultCode}, Intent\emph{ data}}{}
Si el resultado obtenido corresponde al de haber seleccionado una imagen satisfactoriamente. Procede a agregar la imagen como foto de perfil.

\end{fulllineitems}



\subsubsection{setUpResumeField}
\label{Fragments/ProfileFragment:setupresumefield}\index{setUpResumeField(View) (Java method)}

\begin{fulllineitems}
\phantomsection\label{Fragments/ProfileFragment:com.fiuba.tallerii.jobify.ProfileFragment.setUpResumeField(View)}\pysiglinewithargsret{private void \sphinxbfcode{setUpResumeField}}{View\emph{ v}}{}
Prepara el campo del Resumen para que pueda ser editable como corresponde.

\end{fulllineitems}



\subsubsection{setUpNameField}
\label{Fragments/ProfileFragment:setupnamefield}\index{setUpNameField(View) (Java method)}

\begin{fulllineitems}
\phantomsection\label{Fragments/ProfileFragment:com.fiuba.tallerii.jobify.ProfileFragment.setUpNameField(View)}\pysiglinewithargsret{private void \sphinxbfcode{setUpNameField}}{View\emph{ v}}{}
Prepara el campo del Nombre para que pueda ser editable como corresponde.

\end{fulllineitems}



\subsubsection{setUpSaveChangesButton}
\label{Fragments/ProfileFragment:setupsavechangesbutton}\index{setUpSaveChangesButton(View) (Java method)}

\begin{fulllineitems}
\phantomsection\label{Fragments/ProfileFragment:com.fiuba.tallerii.jobify.ProfileFragment.setUpSaveChangesButton(View)}\pysiglinewithargsret{private void \sphinxbfcode{setUpSaveChangesButton}}{View\emph{ v}}{}
Inicializa el botón de guardar los cambios y le agrega los listeners correspondientes.

\end{fulllineitems}



\section{SkillsFragment}
\label{Fragments/SkillsFragment:skillsfragment}\label{Fragments/SkillsFragment::doc}\index{SkillsFragment (Java class)}

\begin{fulllineitems}
\phantomsection\label{Fragments/SkillsFragment:com.fiuba.tallerii.jobify.SkillsFragment}\pysigline{public class \sphinxbfcode{SkillsFragment} extends Fragment}
Utiliza un RecyclerView para enlistar en forma eficiente las destrezas, permitiendo además, la edición de las mismas.

\end{fulllineitems}



\subsection{Fields}
\label{Fragments/SkillsFragment:fields}

\subsubsection{mSkillsRecycleView}
\label{Fragments/SkillsFragment:mskillsrecycleview}\index{mSkillsRecycleView (Java field)}

\begin{fulllineitems}
\phantomsection\label{Fragments/SkillsFragment:com.fiuba.tallerii.jobify.SkillsFragment.mSkillsRecycleView}\pysigline{private RecyclerView \sphinxbfcode{mSkillsRecycleView}}
Referencia al RecyclerView que maneja la vista de las destrzas

\end{fulllineitems}



\subsubsection{mSkillsAdapter}
\label{Fragments/SkillsFragment:mskillsadapter}\index{mSkillsAdapter (Java field)}

\begin{fulllineitems}
\phantomsection\label{Fragments/SkillsFragment:com.fiuba.tallerii.jobify.SkillsFragment.mSkillsAdapter}\pysigline{private {\hyperref[Adapter/SkillsAdapter:com.fiuba.tallerii.jobify.SkillsAdapter]{\sphinxcrossref{SkillsAdapter}}} \sphinxbfcode{mSkillsAdapter}}
Adapter utilizado para manejar la vista de las destrzas en el RecyclerView. Ver \sphinxtitleref{SkillsAdapter}

\end{fulllineitems}



\subsubsection{mAddSkillButton}
\label{Fragments/SkillsFragment:maddskillbutton}\index{mAddSkillButton (Java field)}

\begin{fulllineitems}
\phantomsection\label{Fragments/SkillsFragment:com.fiuba.tallerii.jobify.SkillsFragment.mAddSkillButton}\pysigline{private Button \sphinxbfcode{mAddSkillButton}}
Referencia al botón utilizado para agregar destrezas.

\end{fulllineitems}



\subsection{Methods}
\label{Fragments/SkillsFragment:methods}

\subsubsection{updateUI}
\label{Fragments/SkillsFragment:updateui}\index{updateUI() (Java method)}

\begin{fulllineitems}
\phantomsection\label{Fragments/SkillsFragment:com.fiuba.tallerii.jobify.SkillsFragment.updateUI()}\pysiglinewithargsret{private void \sphinxbfcode{updateUI}}{}{}
Actualiza la vista de las destrezas. Para eso pide la información a \sphinxtitleref{InformationHandler}, luego crea un \sphinxtitleref{SkillsAdapter} y lo asigna al RecyclerView.

\end{fulllineitems}



\subsubsection{onCreateView}
\label{Fragments/SkillsFragment:oncreateview}\index{onCreateView(LayoutInflater, ViewGroup, Bundle) (Java method)}

\begin{fulllineitems}
\phantomsection\label{Fragments/SkillsFragment:com.fiuba.tallerii.jobify.SkillsFragment.onCreateView(LayoutInflater, ViewGroup, Bundle)}\pysiglinewithargsret{public View \sphinxbfcode{onCreateView}}{LayoutInflater\emph{ inflater}, ViewGroup\emph{ container}, Bundle\emph{ savedInstanceState}}{}
Infla el Fragment con su layout correspondiente e inicializa las referencias y componentes.

\end{fulllineitems}



\subsubsection{onResume}
\label{Fragments/SkillsFragment:onresume}\index{onResume() (Java method)}

\begin{fulllineitems}
\phantomsection\label{Fragments/SkillsFragment:com.fiuba.tallerii.jobify.SkillsFragment.onResume()}\pysiglinewithargsret{public void \sphinxbfcode{onResume}}{}{}
Al resumir el Fragment se actualiza la vista de las destrezas invocando a UpdateUI().

\end{fulllineitems}



\section{AddSkillFragment}
\label{Fragments/AddSkillFragment:addskillfragment}\label{Fragments/AddSkillFragment::doc}\index{AddSkillFragment (Java class)}

\begin{fulllineitems}
\phantomsection\label{Fragments/AddSkillFragment:com.fiuba.tallerii.jobify.AddSkillFragment}\pysigline{public class \sphinxbfcode{AddSkillFragment} extends Fragment}
Otorga al usuario una interfaz para agregar un skill a su perfil.

\end{fulllineitems}



\subsection{Fields}
\label{Fragments/AddSkillFragment:fields}

\subsubsection{mTitleEditText}
\label{Fragments/AddSkillFragment:mtitleedittext}

\subsubsection{mCategoryEditText}
\label{Fragments/AddSkillFragment:mcategoryedittext}\index{mCategoryEditText (Java field)}

\begin{fulllineitems}
\phantomsection\label{Fragments/AddSkillFragment:com.fiuba.tallerii.jobify.AddSkillFragment.mCategoryEditText}\pysigline{private EditText \sphinxbfcode{mCategoryEditText}}
Referencia al EditText correspondiente a la forma para ingresar la categoría de la destreza a agregar.

\end{fulllineitems}



\subsubsection{mDescriptionEditText}
\label{Fragments/AddSkillFragment:mdescriptionedittext}\index{mDescriptionEditText (Java field)}

\begin{fulllineitems}
\phantomsection\label{Fragments/AddSkillFragment:com.fiuba.tallerii.jobify.AddSkillFragment.mDescriptionEditText}\pysigline{private EditText \sphinxbfcode{mDescriptionEditText}}
Referencia al EditText correspondiente a la forma para ingresar la descripción de la destreza a agregar.

\end{fulllineitems}



\subsubsection{mAddSkillButton}
\label{Fragments/AddSkillFragment:maddskillbutton}\index{mAddSkillButton (Java field)}

\begin{fulllineitems}
\phantomsection\label{Fragments/AddSkillFragment:com.fiuba.tallerii.jobify.AddSkillFragment.mAddSkillButton}\pysigline{private Button \sphinxbfcode{mAddSkillButton}}
Referencia al botón que confirma los campos ingresados e intenta agregat la nueva destreza al usuario.

\end{fulllineitems}



\subsection{Methods}
\label{Fragments/AddSkillFragment:methods}

\subsubsection{addSkill}
\label{Fragments/AddSkillFragment:addskill}\index{addSkill() (Java method)}

\begin{fulllineitems}
\phantomsection\label{Fragments/AddSkillFragment:com.fiuba.tallerii.jobify.AddSkillFragment.addSkill()}\pysiglinewithargsret{private void \sphinxbfcode{addSkill}}{}{}
Recolecta los datos ingresados en las formas mostradas e inicia una tarea asincrónica para conectarse al servidor y agregar la nueva destreza.

\end{fulllineitems}



\subsubsection{onCreateView}
\label{Fragments/AddSkillFragment:oncreateview}\index{onCreateView(LayoutInflater, ViewGroup, Bundle) (Java method)}

\begin{fulllineitems}
\phantomsection\label{Fragments/AddSkillFragment:com.fiuba.tallerii.jobify.AddSkillFragment.onCreateView(LayoutInflater, ViewGroup, Bundle)}\pysiglinewithargsret{public View \sphinxbfcode{onCreateView}}{LayoutInflater\emph{ inflater}, ViewGroup\emph{ container}, Bundle\emph{ savedInstanceState}}{}
Infla el Fragment con su layout correspondiente e inicializa las referencias y componentes.

\end{fulllineitems}



\section{JobsFragment}
\label{Fragments/JobsFragment:jobsfragment}\label{Fragments/JobsFragment::doc}\index{JobsFragment (Java class)}

\begin{fulllineitems}
\phantomsection\label{Fragments/JobsFragment:com.fiuba.tallerii.jobify.JobsFragment}\pysigline{public class \sphinxbfcode{JobsFragment} extends Fragment}
Utiliza un RecyclerView para enlistar en forma eficiente las experiencias laborales, permitiendo además, la edición de las mismas.

\end{fulllineitems}



\subsection{Fields}
\label{Fragments/JobsFragment:fields}

\subsubsection{mJobsRecycleView}
\label{Fragments/JobsFragment:mjobsrecycleview}\index{mJobsRecycleView (Java field)}

\begin{fulllineitems}
\phantomsection\label{Fragments/JobsFragment:com.fiuba.tallerii.jobify.JobsFragment.mJobsRecycleView}\pysigline{private RecyclerView \sphinxbfcode{mJobsRecycleView}}
Referencia al RecyclerView que maneja la vista de las experiencias laborales

\end{fulllineitems}



\subsubsection{mJobsAdapter}
\label{Fragments/JobsFragment:mjobsadapter}\index{mJobsAdapter (Java field)}

\begin{fulllineitems}
\phantomsection\label{Fragments/JobsFragment:com.fiuba.tallerii.jobify.JobsFragment.mJobsAdapter}\pysigline{private {\hyperref[Adapter/JobsAdapter:com.fiuba.tallerii.jobify.JobsAdapter]{\sphinxcrossref{JobsAdapter}}} \sphinxbfcode{mJobsAdapter}}
Adapter utilizado para manejar la vista de las experiencias laborales en el RecyclerView. Ver \sphinxtitleref{JobsAdapter}

\end{fulllineitems}



\subsubsection{mAddJobButton}
\label{Fragments/JobsFragment:maddjobbutton}\index{mAddJobButton (Java field)}

\begin{fulllineitems}
\phantomsection\label{Fragments/JobsFragment:com.fiuba.tallerii.jobify.JobsFragment.mAddJobButton}\pysigline{private Button \sphinxbfcode{mAddJobButton}}
Referencia al botón utilizado para agregar experiencias laborales

\end{fulllineitems}



\subsection{Methods}
\label{Fragments/JobsFragment:methods}

\subsubsection{updateUI}
\label{Fragments/JobsFragment:updateui}\index{updateUI() (Java method)}

\begin{fulllineitems}
\phantomsection\label{Fragments/JobsFragment:com.fiuba.tallerii.jobify.JobsFragment.updateUI()}\pysiglinewithargsret{private void \sphinxbfcode{updateUI}}{}{}
Actualiza la vista de las experiencias laborales. Para eso pide la información a \sphinxtitleref{InformationHandler}, luego crea un \sphinxtitleref{JobsAdapter} y lo asigna al RecyclerView.

\end{fulllineitems}



\subsubsection{onCreateView}
\label{Fragments/JobsFragment:oncreateview}\index{onCreateView(LayoutInflater, ViewGroup, Bundle) (Java method)}

\begin{fulllineitems}
\phantomsection\label{Fragments/JobsFragment:com.fiuba.tallerii.jobify.JobsFragment.onCreateView(LayoutInflater, ViewGroup, Bundle)}\pysiglinewithargsret{public View \sphinxbfcode{onCreateView}}{LayoutInflater\emph{ inflater}, ViewGroup\emph{ container}, Bundle\emph{ savedInstanceState}}{}
Infla el Fragment con su layout correspondiente e inicializa las referencias y componentes.

\end{fulllineitems}



\subsubsection{onResume}
\label{Fragments/JobsFragment:onresume}\index{onResume() (Java method)}

\begin{fulllineitems}
\phantomsection\label{Fragments/JobsFragment:com.fiuba.tallerii.jobify.JobsFragment.onResume()}\pysiglinewithargsret{public void \sphinxbfcode{onResume}}{}{}
Al resumir el Fragment se actualiza la vista de experiencias laborales.

\end{fulllineitems}



\section{AddJobFragment}
\label{Fragments/AddJobFragment:addjobfragment}\label{Fragments/AddJobFragment::doc}\index{AddJobFragment (Java class)}

\begin{fulllineitems}
\phantomsection\label{Fragments/AddJobFragment:com.fiuba.tallerii.jobify.AddJobFragment}\pysigline{public class \sphinxbfcode{AddJobFragment} extends Fragment}
Otorga al usuario una interfaz para agregar una experiencia laboral a su perfil.

\end{fulllineitems}



\subsection{Fields}
\label{Fragments/AddJobFragment:fields}

\subsubsection{mTitleEditText}
\label{Fragments/AddJobFragment:mtitleedittext}

\subsubsection{mCategoryEditText}
\label{Fragments/AddJobFragment:mcategoryedittext}\index{mCategoryEditText (Java field)}

\begin{fulllineitems}
\phantomsection\label{Fragments/AddJobFragment:com.fiuba.tallerii.jobify.AddJobFragment.mCategoryEditText}\pysigline{private EditText \sphinxbfcode{mCategoryEditText}}
Referencia al EditText correspondiente a la forma para ingresar la categoría de la experiencia laboral a agregar.

\end{fulllineitems}



\subsubsection{mDescriptionEditText}
\label{Fragments/AddJobFragment:mdescriptionedittext}\index{mDescriptionEditText (Java field)}

\begin{fulllineitems}
\phantomsection\label{Fragments/AddJobFragment:com.fiuba.tallerii.jobify.AddJobFragment.mDescriptionEditText}\pysigline{private EditText \sphinxbfcode{mDescriptionEditText}}
Referencia al EditText correspondiente a la forma para ingresar la descripción de la experiencia laboral a agregar.

\end{fulllineitems}



\subsubsection{mAddJobButton}
\label{Fragments/AddJobFragment:maddjobbutton}\index{mAddJobButton (Java field)}

\begin{fulllineitems}
\phantomsection\label{Fragments/AddJobFragment:com.fiuba.tallerii.jobify.AddJobFragment.mAddJobButton}\pysigline{private Button \sphinxbfcode{mAddJobButton}}
Referencia al botón que confirma los campos ingresados e intenta agregat la nueva experiencia laboral al usuario.

\end{fulllineitems}



\subsection{Methods}
\label{Fragments/AddJobFragment:methods}

\subsubsection{addJob}
\label{Fragments/AddJobFragment:addjob}\index{addSkill() (Java method)}

\begin{fulllineitems}
\phantomsection\label{Fragments/AddJobFragment:com.fiuba.tallerii.jobify.AddJobFragment.addSkill()}\pysiglinewithargsret{private void \sphinxbfcode{addSkill}}{}{}
Recolecta los datos ingresados en las formas mostradas e inicia una tarea asincrónica para conectarse al servidor y agregar la nueva experiencia laboral.

\end{fulllineitems}



\subsubsection{onCreateView}
\label{Fragments/AddJobFragment:oncreateview}\index{onCreateView(LayoutInflater, ViewGroup, Bundle) (Java method)}

\begin{fulllineitems}
\phantomsection\label{Fragments/AddJobFragment:com.fiuba.tallerii.jobify.AddJobFragment.onCreateView(LayoutInflater, ViewGroup, Bundle)}\pysiglinewithargsret{public View \sphinxbfcode{onCreateView}}{LayoutInflater\emph{ inflater}, ViewGroup\emph{ container}, Bundle\emph{ savedInstanceState}}{}
Infla el Fragment con su layout correspondiente e inicializa las referencias y componentes.

\end{fulllineitems}



\section{ContactsFragment}
\label{Fragments/ContactsFragment::doc}\label{Fragments/ContactsFragment:contactsfragment}\index{ContactsFragment (Java class)}

\begin{fulllineitems}
\phantomsection\label{Fragments/ContactsFragment:com.fiuba.tallerii.jobify.ContactsFragment}\pysigline{public class \sphinxbfcode{ContactsFragment} extends Fragment}
Utiliza un RecyclerView para enlistar en forma eficiente los contactos del usuario, permitiendo además, la edición de los mismos.

\end{fulllineitems}



\subsection{Fields}
\label{Fragments/ContactsFragment:fields}

\subsubsection{mContactsRecycleView}
\label{Fragments/ContactsFragment:mcontactsrecycleview}\index{mContactsRecycleView (Java field)}

\begin{fulllineitems}
\phantomsection\label{Fragments/ContactsFragment:com.fiuba.tallerii.jobify.ContactsFragment.mContactsRecycleView}\pysigline{private RecyclerView \sphinxbfcode{mContactsRecycleView}}
Referencia al RecyclerView que maneja la vista de los contactos

\end{fulllineitems}



\subsubsection{mContactsAdapter}
\label{Fragments/ContactsFragment:mcontactsadapter}\index{mContactsAdapter (Java field)}

\begin{fulllineitems}
\phantomsection\label{Fragments/ContactsFragment:com.fiuba.tallerii.jobify.ContactsFragment.mContactsAdapter}\pysigline{private {\hyperref[Adapter/ContactsAdapter:com.fiuba.tallerii.jobify.ContactsAdapter]{\sphinxcrossref{ContactsAdapter}}} \sphinxbfcode{mContactsAdapter}}
Adapter utilizado para manejar la vista de los contactos en el RecyclerView. Ver \sphinxtitleref{ContactsAdapter}

\end{fulllineitems}



\subsubsection{mSearchButton}
\label{Fragments/ContactsFragment:msearchbutton}\index{mSearchButton (Java field)}

\begin{fulllineitems}
\phantomsection\label{Fragments/ContactsFragment:com.fiuba.tallerii.jobify.ContactsFragment.mSearchButton}\pysigline{private Button \sphinxbfcode{mSearchButton}}
Referencia al botón utilizado para buscar contactos.

\end{fulllineitems}



\subsection{Methods}
\label{Fragments/ContactsFragment:methods}

\subsubsection{updateUI}
\label{Fragments/ContactsFragment:updateui}\index{startSearchContactActivity() (Java method)}

\begin{fulllineitems}
\phantomsection\label{Fragments/ContactsFragment:com.fiuba.tallerii.jobify.ContactsFragment.startSearchContactActivity()}\pysiglinewithargsret{private void \sphinxbfcode{startSearchContactActivity}}{}{}
Inicia la Activity: \sphinxtitleref{AddContactActivity}.

\end{fulllineitems}



\subsubsection{updateUI}
\label{Fragments/ContactsFragment:id1}\index{updateUI() (Java method)}

\begin{fulllineitems}
\phantomsection\label{Fragments/ContactsFragment:com.fiuba.tallerii.jobify.ContactsFragment.updateUI()}\pysiglinewithargsret{private void \sphinxbfcode{updateUI}}{}{}
Actualiza la vista de los contactos en el RecyclerView. Para eso pide la información a \sphinxtitleref{InformationHandler}, luego crea un \sphinxtitleref{ContactsAdapter} y lo asigna al RecyclerView.

\end{fulllineitems}



\subsubsection{onCreateView}
\label{Fragments/ContactsFragment:oncreateview}\index{onCreateView(LayoutInflater, ViewGroup, Bundle) (Java method)}

\begin{fulllineitems}
\phantomsection\label{Fragments/ContactsFragment:com.fiuba.tallerii.jobify.ContactsFragment.onCreateView(LayoutInflater, ViewGroup, Bundle)}\pysiglinewithargsret{public View \sphinxbfcode{onCreateView}}{LayoutInflater\emph{ inflater}, ViewGroup\emph{ container}, Bundle\emph{ savedInstanceState}}{}
Infla el Fragment con su layout correspondiente e inicializa las referencias y componentes.

\end{fulllineitems}



\subsubsection{onResume}
\label{Fragments/ContactsFragment:onresume}\index{onResume() (Java method)}

\begin{fulllineitems}
\phantomsection\label{Fragments/ContactsFragment:com.fiuba.tallerii.jobify.ContactsFragment.onResume()}\pysiglinewithargsret{public void \sphinxbfcode{onResume}}{}{}
Al resumir el Fragment se actualiza la vista de los contactos invocando a UpdateUI().

\end{fulllineitems}



\section{AddContactFragment}
\label{Fragments/AddContactFragment::doc}\label{Fragments/AddContactFragment:addcontactfragment}\index{AddContactFragment (Java class)}

\begin{fulllineitems}
\phantomsection\label{Fragments/AddContactFragment:com.fiuba.tallerii.jobify.AddContactFragment}\pysigline{public class \sphinxbfcode{AddContactFragment} extends Fragment}
Otorga al usuario una interfaz para agregar una experiencia laboral a su perfil.

\end{fulllineitems}



\subsection{Fields}
\label{Fragments/AddContactFragment:fields}

\subsubsection{mUsernameEditText}
\label{Fragments/AddContactFragment:musernameedittext}\index{mUsernameEditText (Java field)}

\begin{fulllineitems}
\phantomsection\label{Fragments/AddContactFragment:com.fiuba.tallerii.jobify.AddContactFragment.mUsernameEditText}\pysigline{private EditText \sphinxbfcode{mUsernameEditText}}
Referencia al EditText correspondiente a la forma para buscar un contacto.

\end{fulllineitems}



\subsubsection{mAddContactButton}
\label{Fragments/AddContactFragment:maddcontactbutton}\index{mAddContactButton (Java field)}

\begin{fulllineitems}
\phantomsection\label{Fragments/AddContactFragment:com.fiuba.tallerii.jobify.AddContactFragment.mAddContactButton}\pysigline{private Button \sphinxbfcode{mAddContactButton}}
Referencia al botón que envía una solicitud de amistad al contacto seleccionado, para esperar ser aceptado.

\end{fulllineitems}



\subsection{Methods}
\label{Fragments/AddContactFragment:methods}

\subsubsection{addContact}
\label{Fragments/AddContactFragment:addcontact}\index{addContact() (Java method)}

\begin{fulllineitems}
\phantomsection\label{Fragments/AddContactFragment:com.fiuba.tallerii.jobify.AddContactFragment.addContact()}\pysiglinewithargsret{private void \sphinxbfcode{addContact}}{}{}
Inicia una tarea asincrónica para conectarse al servidor y agregar al contacto seleccionado.

\end{fulllineitems}



\subsubsection{isUsernameValid}
\label{Fragments/AddContactFragment:isusernamevalid}\index{isUsernameValid(String) (Java method)}

\begin{fulllineitems}
\phantomsection\label{Fragments/AddContactFragment:com.fiuba.tallerii.jobify.AddContactFragment.isUsernameValid(String)}\pysiglinewithargsret{private boolean \sphinxbfcode{isUsernameValid}}{\href{http://docs.oracle.com/javase/6/docs/api/java/lang/String.html}{String}\emph{ username}}{}
Devuelve true si el usuario ingresado por parámetro existe en la base de datos del servidor. False en caso contrario.
\begin{quote}\begin{description}
\item[{Parameters}] \leavevmode\begin{itemize}
\item {} 
\textbf{\texttt{username}} -- nombre de usuario del contacto a agregar

\end{itemize}

\end{description}\end{quote}

\end{fulllineitems}



\subsubsection{onCreateView}
\label{Fragments/AddContactFragment:oncreateview}\index{onCreateView(LayoutInflater, ViewGroup, Bundle) (Java method)}

\begin{fulllineitems}
\phantomsection\label{Fragments/AddContactFragment:com.fiuba.tallerii.jobify.AddContactFragment.onCreateView(LayoutInflater, ViewGroup, Bundle)}\pysiglinewithargsret{public View \sphinxbfcode{onCreateView}}{LayoutInflater\emph{ inflater}, ViewGroup\emph{ container}, Bundle\emph{ savedInstanceState}}{}
Infla el Fragment con su layout correspondiente e inicializa las referencias y componentes.

\end{fulllineitems}



\section{NotificationsFragment}
\label{Fragments/NotificationsFragment:notificationsfragment}\label{Fragments/NotificationsFragment::doc}\index{NotificationsFragment (Java class)}

\begin{fulllineitems}
\phantomsection\label{Fragments/NotificationsFragment:com.fiuba.tallerii.jobify.NotificationsFragment}\pysigline{public class \sphinxbfcode{NotificationsFragment} extends Fragment}
Utiliza un RecyclerView para enlistar en forma eficiente las destrezas, permitiendo además, la edición de las mismas.

\end{fulllineitems}



\subsection{Fields}
\label{Fragments/NotificationsFragment:fields}

\subsubsection{mNotificationsRecycleView}
\label{Fragments/NotificationsFragment:mnotificationsrecycleview}\index{mNotificationsRecycleView (Java field)}

\begin{fulllineitems}
\phantomsection\label{Fragments/NotificationsFragment:com.fiuba.tallerii.jobify.NotificationsFragment.mNotificationsRecycleView}\pysigline{private RecyclerView \sphinxbfcode{mNotificationsRecycleView}}
Referencia al RecyclerView que maneja la vista de las notificaciones.

\end{fulllineitems}



\subsubsection{mNotificationsAdapter}
\label{Fragments/NotificationsFragment:mnotificationsadapter}\index{mNotificationsAdapter (Java field)}

\begin{fulllineitems}
\phantomsection\label{Fragments/NotificationsFragment:com.fiuba.tallerii.jobify.NotificationsFragment.mNotificationsAdapter}\pysigline{private NotificationsAdapter \sphinxbfcode{mNotificationsAdapter}}
Adapter utilizado para manejar la vista de las notificaciones en el RecyclerView. Ver \sphinxtitleref{NotificationsAdapter}.

\end{fulllineitems}



\subsubsection{mAddingFriend}
\label{Fragments/NotificationsFragment:maddingfriend}\index{mAddingFriend (Java field)}

\begin{fulllineitems}
\phantomsection\label{Fragments/NotificationsFragment:com.fiuba.tallerii.jobify.NotificationsFragment.mAddingFriend}\pysigline{private boolean \sphinxbfcode{mAddingFriend} = false}
Define si se verán o no los títulos de los tabs. Por defecto, los títulos no se verán.

\end{fulllineitems}



\subsection{Methods}
\label{Fragments/NotificationsFragment:methods}

\subsubsection{createAddFriendDialog}
\label{Fragments/NotificationsFragment:createaddfrienddialog}\index{createAddFriendDialog(String, int) (Java method)}

\begin{fulllineitems}
\phantomsection\label{Fragments/NotificationsFragment:com.fiuba.tallerii.jobify.NotificationsFragment.createAddFriendDialog(String, int)}\pysiglinewithargsret{private void \sphinxbfcode{createAddFriendDialog}}{\href{http://docs.oracle.com/javase/6/docs/api/java/lang/String.html}{String}\emph{ friendUsername}, int\emph{ notificationIndex}}{}
Crea un Dialog en respuesta a una notificación de solicitud de amistad, en el cual se puede aceptar o rechazar al contacto solicitante.
\begin{quote}\begin{description}
\item[{Parameters}] \leavevmode\begin{itemize}
\item {} 
\textbf{\texttt{friendUsername}} -- Nombre de usuario del usuario que envío la solicitud.

\item {} 
\textbf{\texttt{notificationIndex}} -- índice correspondiente a la lista de notificaciones.

\end{itemize}

\end{description}\end{quote}

\end{fulllineitems}



\subsubsection{updateUI}
\label{Fragments/NotificationsFragment:updateui}\index{updateUI() (Java method)}

\begin{fulllineitems}
\phantomsection\label{Fragments/NotificationsFragment:com.fiuba.tallerii.jobify.NotificationsFragment.updateUI()}\pysiglinewithargsret{private void \sphinxbfcode{updateUI}}{}{}
Actualiza la vista de las notificaciones. Para eso pide la información a \sphinxtitleref{InformationHandler}, luego crea un \sphinxtitleref{NotificationsAdapter} y lo asigna al RecyclerView.

\end{fulllineitems}



\subsubsection{onCreate}
\label{Fragments/NotificationsFragment:oncreate}\index{onCreateView(LayoutInflater, ViewGroup, Bundle) (Java method)}

\begin{fulllineitems}
\phantomsection\label{Fragments/NotificationsFragment:com.fiuba.tallerii.jobify.NotificationsFragment.onCreateView(LayoutInflater, ViewGroup, Bundle)}\pysiglinewithargsret{public View \sphinxbfcode{onCreateView}}{LayoutInflater\emph{ inflater}, ViewGroup\emph{ container}, Bundle\emph{ savedInstanceState}}{}
Infla el Fragment con su layout correspondiente e inicializa las referencias y componentes.

\end{fulllineitems}



\subsubsection{onResume}
\label{Fragments/NotificationsFragment:onresume}\index{onResume() (Java method)}

\begin{fulllineitems}
\phantomsection\label{Fragments/NotificationsFragment:com.fiuba.tallerii.jobify.NotificationsFragment.onResume()}\pysiglinewithargsret{public void \sphinxbfcode{onResume}}{}{}
Al resumir el Fragment se actualiza la vista de las notificaciones invocando a UpdateUI().

\end{fulllineitems}



\chapter{Singletons}
\label{Singletons/package-index:singletons}\label{Singletons/package-index::doc}

\section{ServerHandler}
\label{Singletons/ServerHandler:serverhandler}\label{Singletons/ServerHandler::doc}\index{ServerHandler (Java class)}

\begin{fulllineitems}
\phantomsection\label{Singletons/ServerHandler:com.fiuba.tallerii.jobify.ServerHandler}\pysigline{public class \sphinxbfcode{ServerHandler}}
Permite realizar operaciones HTTP a través de una API Rest, utilizando un token de authenticación.
Es importante mencionar, que estas operaciones NO deben realizarse en la UI thread de Android, sino que deben ser desplazadas a otra thread. En esta aplicación, se utilizan AsyncTasks para ejecutar las operaciones del ServerHandler en la Background thread.

\end{fulllineitems}



\subsection{Fields}
\label{Singletons/ServerHandler:fields}

\subsubsection{mServerIP}
\label{Singletons/ServerHandler:mserverip}\index{mServerIP (Java field)}

\begin{fulllineitems}
\phantomsection\label{Singletons/ServerHandler:com.fiuba.tallerii.jobify.ServerHandler.mServerIP}\pysigline{private \href{http://docs.oracle.com/javase/6/docs/api/java/lang/String.html}{String} \sphinxbfcode{mServerIP}}
IP del server con puerto incluido, el cual utilizarña para conectarse al servidor. Ejemplo: 192.168.0.19:8000

\end{fulllineitems}



\subsubsection{mUsername}
\label{Singletons/ServerHandler:musername}\index{mUsername (Java field)}

\begin{fulllineitems}
\phantomsection\label{Singletons/ServerHandler:com.fiuba.tallerii.jobify.ServerHandler.mUsername}\pysigline{private \href{http://docs.oracle.com/javase/6/docs/api/java/lang/String.html}{String} \sphinxbfcode{mUsername}}
Nombre de usuario con el cual se ha establecido la conexión

\end{fulllineitems}



\subsubsection{mConnectionToken}
\label{Singletons/ServerHandler:mconnectiontoken}\index{mConnectionToken (Java field)}

\begin{fulllineitems}
\phantomsection\label{Singletons/ServerHandler:com.fiuba.tallerii.jobify.ServerHandler.mConnectionToken}\pysigline{private \href{http://docs.oracle.com/javase/6/docs/api/java/lang/String.html}{String} \sphinxbfcode{mConnectionToken}}
Token provista para el usuario actual.

\end{fulllineitems}



\subsection{Methods}
\label{Singletons/ServerHandler:methods}

\subsubsection{get}
\label{Singletons/ServerHandler:get}\index{get(Context) (Java method)}

\begin{fulllineitems}
\phantomsection\label{Singletons/ServerHandler:com.fiuba.tallerii.jobify.ServerHandler.get(Context)}\pysiglinewithargsret{public static {\hyperref[Singletons/ServerHandler:com.fiuba.tallerii.jobify.ServerHandler]{\sphinxcrossref{ServerHandler}}} \sphinxbfcode{get}}{Context\emph{ context}}{}
Devuelve la instancia actual del Singleton ServerHandler.

\end{fulllineitems}



\subsubsection{setConnectionToken}
\label{Singletons/ServerHandler:setconnectiontoken}\index{setConnectionToken(String) (Java method)}

\begin{fulllineitems}
\phantomsection\label{Singletons/ServerHandler:com.fiuba.tallerii.jobify.ServerHandler.setConnectionToken(String)}\pysiglinewithargsret{public void \sphinxbfcode{setConnectionToken}}{\href{http://docs.oracle.com/javase/6/docs/api/java/lang/String.html}{String}\emph{ connectionToken}}{}
Permite utilizar el token provisto en los headers de los mensajes HTTP realizados.
\begin{quote}\begin{description}
\item[{Parameters}] \leavevmode\begin{itemize}
\item {} 
\textbf{\texttt{connectionToken}} -- token de authenticación provisto

\end{itemize}

\end{description}\end{quote}

\end{fulllineitems}



\subsubsection{GET}
\label{Singletons/ServerHandler:id1}\index{GET(String) (Java method)}

\begin{fulllineitems}
\phantomsection\label{Singletons/ServerHandler:com.fiuba.tallerii.jobify.ServerHandler.GET(String)}\pysiglinewithargsret{public \href{http://docs.oracle.com/javase/6/docs/api/java/lang/String.html}{String} \sphinxbfcode{GET}}{\href{http://docs.oracle.com/javase/6/docs/api/java/lang/String.html}{String}\emph{ urlSpec}}{}
Realiza un GET request a la dirección especificada por el parámetro urlSpec.
\begin{quote}\begin{description}
\item[{Parameters}] \leavevmode\begin{itemize}
\item {} 
\textbf{\texttt{urlSpec}} -- URL a en la que se establecerá la conexión

\end{itemize}

\item[{Returns}] \leavevmode
Respuesta del server

\end{description}\end{quote}

\end{fulllineitems}



\subsubsection{POST}
\label{Singletons/ServerHandler:post}\index{POST(String, String) (Java method)}

\begin{fulllineitems}
\phantomsection\label{Singletons/ServerHandler:com.fiuba.tallerii.jobify.ServerHandler.POST(String, String)}\pysiglinewithargsret{public \href{http://docs.oracle.com/javase/6/docs/api/java/lang/String.html}{String} \sphinxbfcode{POST}}{\href{http://docs.oracle.com/javase/6/docs/api/java/lang/String.html}{String}\emph{ urlSpec}, \href{http://docs.oracle.com/javase/6/docs/api/java/lang/String.html}{String}\emph{ parameters}}{}
Realiza un POST request a la dirección especificada por el parámetro urlSpec.
\begin{quote}\begin{description}
\item[{Parameters}] \leavevmode\begin{itemize}
\item {} 
\textbf{\texttt{urlSpec}} -- URL a en la que se establecerá la conexión

\item {} 
\textbf{\texttt{parameters}} -- parametros del POST en formato json

\end{itemize}

\item[{Returns}] \leavevmode
Respuesta del server

\end{description}\end{quote}

\end{fulllineitems}



\subsubsection{PUT}
\label{Singletons/ServerHandler:put}\index{PUT(String, String) (Java method)}

\begin{fulllineitems}
\phantomsection\label{Singletons/ServerHandler:com.fiuba.tallerii.jobify.ServerHandler.PUT(String, String)}\pysiglinewithargsret{public \href{http://docs.oracle.com/javase/6/docs/api/java/lang/String.html}{String} \sphinxbfcode{PUT}}{\href{http://docs.oracle.com/javase/6/docs/api/java/lang/String.html}{String}\emph{ urlSpec}, \href{http://docs.oracle.com/javase/6/docs/api/java/lang/String.html}{String}\emph{ parameters}}{}
Realiza un PUT request a la dirección especificada por el parámetro urlSpec.
\begin{quote}\begin{description}
\item[{Parameters}] \leavevmode\begin{itemize}
\item {} 
\textbf{\texttt{urlSpec}} -- URL a en la que se establecerá la conexión

\item {} 
\textbf{\texttt{parameters}} -- parametros del PUT en formato json

\end{itemize}

\item[{Returns}] \leavevmode
Respuesta del server

\end{description}\end{quote}

\end{fulllineitems}



\subsubsection{DELETE}
\label{Singletons/ServerHandler:delete}\index{DELETE(String, String) (Java method)}

\begin{fulllineitems}
\phantomsection\label{Singletons/ServerHandler:com.fiuba.tallerii.jobify.ServerHandler.DELETE(String, String)}\pysiglinewithargsret{public \href{http://docs.oracle.com/javase/6/docs/api/java/lang/String.html}{String} \sphinxbfcode{DELETE}}{\href{http://docs.oracle.com/javase/6/docs/api/java/lang/String.html}{String}\emph{ urlSpec}, \href{http://docs.oracle.com/javase/6/docs/api/java/lang/String.html}{String}\emph{ parameters}}{}
Realiza un DELETE request a la dirección especificada por el parámetro urlSpec.
\begin{quote}\begin{description}
\item[{Parameters}] \leavevmode\begin{itemize}
\item {} 
\textbf{\texttt{urlSpec}} -- URL a en la que se establecerá la conexión

\item {} 
\textbf{\texttt{parameters}} -- parametros del DELETE en formato json

\end{itemize}

\item[{Returns}] \leavevmode
Respuesta del server

\end{description}\end{quote}

\end{fulllineitems}



\section{InformationHolder}
\label{Singletons/InformationHolder::doc}\label{Singletons/InformationHolder:informationholder}\index{InformationHolder (Java class)}

\begin{fulllineitems}
\phantomsection\label{Singletons/InformationHolder:com.fiuba.tallerii.jobify.InformationHolder}\pysigline{public class \sphinxbfcode{InformationHolder}}
Permite realizar operaciones HTTP a través de una API Rest, utilizando un token de authenticación.
Es importante mencionar, que estas operaciones NO deben realizarse en la UI thread de Android, sino que deben ser desplazadas a otra thread. En esta aplicación, se utilizan AsyncTasks para ejecutar las operaciones del InformationHolder en la Background thread.

\end{fulllineitems}



\subsection{Fields}
\label{Singletons/InformationHolder:fields}

\subsubsection{mName}
\label{Singletons/InformationHolder:mname}\index{mName (Java field)}

\begin{fulllineitems}
\phantomsection\label{Singletons/InformationHolder:com.fiuba.tallerii.jobify.InformationHolder.mName}\pysigline{private \href{http://docs.oracle.com/javase/6/docs/api/java/lang/String.html}{String} \sphinxbfcode{mName}}
Nombre del usuario actualmente conectado.

\end{fulllineitems}



\subsubsection{mResume}
\label{Singletons/InformationHolder:mresume}\index{mResume (Java field)}

\begin{fulllineitems}
\phantomsection\label{Singletons/InformationHolder:com.fiuba.tallerii.jobify.InformationHolder.mResume}\pysigline{private \href{http://docs.oracle.com/javase/6/docs/api/java/lang/String.html}{String} \sphinxbfcode{mResume}}
Resumen del usuario actual

\end{fulllineitems}



\subsubsection{mMail}
\label{Singletons/InformationHolder:mmail}\index{mMail (Java field)}

\begin{fulllineitems}
\phantomsection\label{Singletons/InformationHolder:com.fiuba.tallerii.jobify.InformationHolder.mMail}\pysigline{private \href{http://docs.oracle.com/javase/6/docs/api/java/lang/String.html}{String} \sphinxbfcode{mMail}}
Email del usuario actual

\end{fulllineitems}



\subsubsection{mProfilePicture}
\label{Singletons/InformationHolder:mprofilepicture}\index{mProfilePicture (Java field)}

\begin{fulllineitems}
\phantomsection\label{Singletons/InformationHolder:com.fiuba.tallerii.jobify.InformationHolder.mProfilePicture}\pysigline{private Bitmap \sphinxbfcode{mProfilePicture}}
Bitmap correspondiente a la foto de perfil del usuario actual.

\end{fulllineitems}



\subsubsection{mContacts}
\label{Singletons/InformationHolder:mcontacts}\index{mContacts (Java field)}

\begin{fulllineitems}
\phantomsection\label{Singletons/InformationHolder:com.fiuba.tallerii.jobify.InformationHolder.mContacts}\pysigline{private List\textless{}{\hyperref[Model/Contact:com.fiuba.tallerii.jobify.Contact]{\sphinxcrossref{Contact}}}\textgreater{} \sphinxbfcode{mContacts}}
Lista de contactos del usuario actual

\end{fulllineitems}



\subsubsection{mJobs}
\label{Singletons/InformationHolder:mjobs}\index{mJobs (Java field)}

\begin{fulllineitems}
\phantomsection\label{Singletons/InformationHolder:com.fiuba.tallerii.jobify.InformationHolder.mJobs}\pysigline{private List\textless{}{\hyperref[Model/Job:com.fiuba.tallerii.jobify.Job]{\sphinxcrossref{Job}}}\textgreater{} \sphinxbfcode{mJobs}}
Lista de experiencias laborales del usuario actual

\end{fulllineitems}



\subsubsection{mSkills}
\label{Singletons/InformationHolder:mskills}\index{mSkills (Java field)}

\begin{fulllineitems}
\phantomsection\label{Singletons/InformationHolder:com.fiuba.tallerii.jobify.InformationHolder.mSkills}\pysigline{private List\textless{}{\hyperref[Model/Skill:com.fiuba.tallerii.jobify.Skill]{\sphinxcrossref{Skill}}}\textgreater{} \sphinxbfcode{mSkills}}
Lista de destrezas del usuario actual

\end{fulllineitems}



\subsubsection{mNotifications}
\label{Singletons/InformationHolder:mnotifications}\index{mNotifications (Java field)}

\begin{fulllineitems}
\phantomsection\label{Singletons/InformationHolder:com.fiuba.tallerii.jobify.InformationHolder.mNotifications}\pysigline{private List\textless{}{\hyperref[Model/Notification:com.fiuba.tallerii.jobify.Notification]{\sphinxcrossref{Notification}}}\textgreater{} \sphinxbfcode{mNotifications}}
Lista de notificaciones del usuario actual

\end{fulllineitems}



\subsection{Methods}
\label{Singletons/InformationHolder:methods}
Getters y Setters de los atributos anteriores.


\subsubsection{get}
\label{Singletons/InformationHolder:get}\index{get() (Java method)}

\begin{fulllineitems}
\phantomsection\label{Singletons/InformationHolder:com.fiuba.tallerii.jobify.InformationHolder.get()}\pysiglinewithargsret{public static {\hyperref[Singletons/InformationHolder:com.fiuba.tallerii.jobify.InformationHolder]{\sphinxcrossref{InformationHolder}}} \sphinxbfcode{get}}{}{}
Devuelve la instancia actual del Singleton InformationHolder.

\end{fulllineitems}



\chapter{Adapters}
\label{Adapter/package-index:adapters}\label{Adapter/package-index::doc}

\section{ViewPagerAdapter}
\label{Adapter/ViewPagerAdapter::doc}\label{Adapter/ViewPagerAdapter:viewpageradapter}\index{ViewPagerAdapter (Java class)}

\begin{fulllineitems}
\phantomsection\label{Adapter/ViewPagerAdapter:com.fiuba.tallerii.jobify.ViewPagerAdapter}\pysigline{public class \sphinxbfcode{ViewPagerAdapter} extends FragmentPagerAdapter}
Adaptador definido internamente por \sphinxtitleref{MainScreenActivity} para controlar los tabs.

\end{fulllineitems}



\subsection{Fields}
\label{Adapter/ViewPagerAdapter:fields}

\subsubsection{mFragmentList}
\label{Adapter/ViewPagerAdapter:mfragmentlist}\index{mFragmentList (Java field)}

\begin{fulllineitems}
\phantomsection\label{Adapter/ViewPagerAdapter:com.fiuba.tallerii.jobify.ViewPagerAdapter.mFragmentList}\pysigline{private final List\textless{}Fragment\textgreater{} \sphinxbfcode{mFragmentList}}
Lista constante de Fragments que serán manipulados

\end{fulllineitems}



\subsubsection{mFragmentTitleList}
\label{Adapter/ViewPagerAdapter:mfragmenttitlelist}\index{mFragmentTitleList (Java field)}

\begin{fulllineitems}
\phantomsection\label{Adapter/ViewPagerAdapter:com.fiuba.tallerii.jobify.ViewPagerAdapter.mFragmentTitleList}\pysigline{private final List\textless{}\href{http://docs.oracle.com/javase/6/docs/api/java/lang/String.html}{String}\textgreater{} \sphinxbfcode{mFragmentTitleList}}
Lista constante con los títulos de los Fragments.

\end{fulllineitems}



\subsubsection{mShowTittle}
\label{Adapter/ViewPagerAdapter:mshowtittle}\index{mShowTittle (Java field)}

\begin{fulllineitems}
\phantomsection\label{Adapter/ViewPagerAdapter:com.fiuba.tallerii.jobify.ViewPagerAdapter.mShowTittle}\pysigline{private boolean \sphinxbfcode{mShowTittle}}
Determina si el usuario verá el título de cada tab o si solo verá los íconos.

\end{fulllineitems}



\subsection{Constructor}
\label{Adapter/ViewPagerAdapter:constructor}

\subsubsection{ViewPagerAdapter}
\label{Adapter/ViewPagerAdapter:id1}\index{ViewPagerAdapter(FragmentManager) (Java constructor)}

\begin{fulllineitems}
\phantomsection\label{Adapter/ViewPagerAdapter:com.fiuba.tallerii.jobify.ViewPagerAdapter.ViewPagerAdapter(FragmentManager)}\pysiglinewithargsret{public \sphinxbfcode{ViewPagerAdapter}}{FragmentManager\emph{ manager}}{}
Contructor, inicializa el Adapter y setea por defecto que no se vean los nombre de los tabs.
\begin{quote}\begin{description}
\item[{Parameters}] \leavevmode\begin{itemize}
\item {} 
\textbf{\texttt{manager}} -- 

\end{itemize}

\end{description}\end{quote}

\end{fulllineitems}



\subsection{Methods}
\label{Adapter/ViewPagerAdapter:methods}

\subsubsection{addFragment}
\label{Adapter/ViewPagerAdapter:addfragment}\index{addFragment(Fragment, String) (Java method)}

\begin{fulllineitems}
\phantomsection\label{Adapter/ViewPagerAdapter:com.fiuba.tallerii.jobify.ViewPagerAdapter.addFragment(Fragment, String)}\pysiglinewithargsret{public void \sphinxbfcode{addFragment}}{Fragment\emph{ fragment}, \href{http://docs.oracle.com/javase/6/docs/api/java/lang/String.html}{String}\emph{ title}}{}
Agrega un Fragment con el título pasado como segundo parámetro, para ser manipulado por el ViewPagerAdapter.
\begin{quote}\begin{description}
\item[{Parameters}] \leavevmode\begin{itemize}
\item {} 
\textbf{\texttt{fragment}} -- Fragment a manipular

\item {} 
\textbf{\texttt{tittle}} -- títuló asociado al Fragment

\end{itemize}

\end{description}\end{quote}

\end{fulllineitems}



\subsubsection{getItem}
\label{Adapter/ViewPagerAdapter:getitem}\index{getItem(int) (Java method)}

\begin{fulllineitems}
\phantomsection\label{Adapter/ViewPagerAdapter:com.fiuba.tallerii.jobify.ViewPagerAdapter.getItem(int)}\pysiglinewithargsret{public Fragment \sphinxbfcode{getItem}}{int\emph{ position}}{}
Devuelve el Fragment en la posicion pasada como parámetro
\begin{quote}\begin{description}
\item[{Parameters}] \leavevmode\begin{itemize}
\item {} 
\textbf{\texttt{position}} -- índice del Fragment en la lista.

\end{itemize}

\end{description}\end{quote}

\end{fulllineitems}



\subsubsection{getCount}
\label{Adapter/ViewPagerAdapter:getcount}\index{getCount() (Java method)}

\begin{fulllineitems}
\phantomsection\label{Adapter/ViewPagerAdapter:com.fiuba.tallerii.jobify.ViewPagerAdapter.getCount()}\pysiglinewithargsret{public int \sphinxbfcode{getCount}}{}{}
Devuelve la cantidad de Fragments que están siendo manipulados por la ViewPagerAdapter

\end{fulllineitems}



\subsubsection{getPageTitle}
\label{Adapter/ViewPagerAdapter:getpagetitle}\index{getPageTitle(int) (Java method)}

\begin{fulllineitems}
\phantomsection\label{Adapter/ViewPagerAdapter:com.fiuba.tallerii.jobify.ViewPagerAdapter.getPageTitle(int)}\pysiglinewithargsret{public \href{http://docs.oracle.com/javase/6/docs/api/java/lang/CharSequence.html}{CharSequence} \sphinxbfcode{getPageTitle}}{int\emph{ position}}{}
Devuelve el título del Fragment correspondiente al índice pasado por parámetro
\begin{quote}\begin{description}
\item[{Parameters}] \leavevmode\begin{itemize}
\item {} 
\textbf{\texttt{position}} -- índice del Fragment en la lista.

\end{itemize}

\end{description}\end{quote}

\end{fulllineitems}



\section{SkillsAdapter}
\label{Adapter/SkillsAdapter::doc}\label{Adapter/SkillsAdapter:skillsadapter}\index{SkillsAdapter (Java class)}

\begin{fulllineitems}
\phantomsection\label{Adapter/SkillsAdapter:com.fiuba.tallerii.jobify.SkillsAdapter}\pysigline{private class \sphinxbfcode{SkillsAdapter} extends RecyclerView.Adapter\textless{}SkillsViewHolder\textgreater{}}
Adaptador definido internamente por \sphinxtitleref{SkillsFragment} para controlar cada destreza en forma de un \sphinxtitleref{SkillsViewHolder}

\end{fulllineitems}



\subsection{Fields}
\label{Adapter/SkillsAdapter:fields}

\subsubsection{mSkills}
\label{Adapter/SkillsAdapter:mskills}\index{mSkills (Java field)}

\begin{fulllineitems}
\phantomsection\label{Adapter/SkillsAdapter:com.fiuba.tallerii.jobify.SkillsAdapter.mSkills}\pysigline{private List\textless{}{\hyperref[Model/Skill:com.fiuba.tallerii.jobify.Skill]{\sphinxcrossref{Skill}}}\textgreater{} \sphinxbfcode{mSkills}}
Lista de \sphinxtitleref{Skill}, que contienen la información acerca de las destrezas del usuario.

\end{fulllineitems}



\subsection{Constructor}
\label{Adapter/SkillsAdapter:constructor}

\subsubsection{SkillsAdapter}
\label{Adapter/SkillsAdapter:id1}\index{SkillsAdapter(List) (Java constructor)}

\begin{fulllineitems}
\phantomsection\label{Adapter/SkillsAdapter:com.fiuba.tallerii.jobify.SkillsAdapter.SkillsAdapter(List)}\pysiglinewithargsret{public \sphinxbfcode{SkillsAdapter}}{List\textless{}{\hyperref[Model/Skill:com.fiuba.tallerii.jobify.Skill]{\sphinxcrossref{Skill}}}\textgreater{}\emph{ skills}}{}
Inicializa la lista de \sphinxtitleref{Skill} a manipular.
\begin{quote}\begin{description}
\item[{Parameters}] \leavevmode\begin{itemize}
\item {} 
\textbf{\texttt{skills}} -- 

\end{itemize}

\end{description}\end{quote}

\end{fulllineitems}



\subsection{Methods}
\label{Adapter/SkillsAdapter:methods}

\subsubsection{onCreateViewHolder}
\label{Adapter/SkillsAdapter:oncreateviewholder}\index{onCreateViewHolder(ViewGroup, int) (Java method)}

\begin{fulllineitems}
\phantomsection\label{Adapter/SkillsAdapter:com.fiuba.tallerii.jobify.SkillsAdapter.onCreateViewHolder(ViewGroup, int)}\pysiglinewithargsret{public NotificationsViewHolder \sphinxbfcode{onCreateViewHolder}}{ViewGroup\emph{ parent}, int\emph{ viewType}}{}
Crea el View para mostrar un \sphinxtitleref{Skill} de la forma deseada y crea con el mismo, un {\color{red}\bfseries{}{}`}SkillsViewHolder'
\begin{quote}\begin{description}
\item[{Parameters}] \leavevmode\begin{itemize}
\item {} 
\textbf{\texttt{parent}} -- 

\item {} 
\textbf{\texttt{viewType}} -- 

\end{itemize}

\end{description}\end{quote}

\end{fulllineitems}



\subsubsection{getItemCount}
\label{Adapter/SkillsAdapter:getitemcount}\index{getItemCount() (Java method)}

\begin{fulllineitems}
\phantomsection\label{Adapter/SkillsAdapter:com.fiuba.tallerii.jobify.SkillsAdapter.getItemCount()}\pysiglinewithargsret{public int \sphinxbfcode{getItemCount}}{}{}
Devuelve la cantidad de Skill's que se están manipulando

\end{fulllineitems}



\subsubsection{onBindViewHolder}
\label{Adapter/SkillsAdapter:onbindviewholder}\index{onBindViewHolder(SkillsViewHolder, int) (Java method)}

\begin{fulllineitems}
\phantomsection\label{Adapter/SkillsAdapter:com.fiuba.tallerii.jobify.SkillsAdapter.onBindViewHolder(SkillsViewHolder, int)}\pysiglinewithargsret{public void \sphinxbfcode{onBindViewHolder}}{SkillsViewHolder\emph{ holder}, int\emph{ position}}{}
Rellena los valores del SkillsViewHolder para que pueda ser mostrado de la forma deseada.
\begin{quote}\begin{description}
\item[{Parameters}] \leavevmode\begin{itemize}
\item {} 
\textbf{\texttt{holder}} -- SkillsViewHolder que se está agregando

\item {} 
\textbf{\texttt{position}} -- índice en la lista de Skills, del \sphinxtitleref{Skill} que está siendo agregado.

\end{itemize}

\end{description}\end{quote}

\end{fulllineitems}



\section{JobsAdapter}
\label{Adapter/JobsAdapter::doc}\label{Adapter/JobsAdapter:jobsadapter}\index{JobsAdapter (Java class)}

\begin{fulllineitems}
\phantomsection\label{Adapter/JobsAdapter:com.fiuba.tallerii.jobify.JobsAdapter}\pysigline{private class \sphinxbfcode{JobsAdapter} extends RecyclerView.Adapter\textless{}JobsViewHolder\textgreater{}}
Adaptador definido internamente por \sphinxtitleref{JobsFragment} para controlar cada experiencia laboral en forma de un \sphinxtitleref{JobsViewHolder}

\end{fulllineitems}



\subsection{Fields}
\label{Adapter/JobsAdapter:fields}

\subsubsection{mJobs}
\label{Adapter/JobsAdapter:mjobs}\index{mJobs (Java field)}

\begin{fulllineitems}
\phantomsection\label{Adapter/JobsAdapter:com.fiuba.tallerii.jobify.JobsAdapter.mJobs}\pysigline{private List\textless{}{\hyperref[Model/Job:com.fiuba.tallerii.jobify.Job]{\sphinxcrossref{Job}}}\textgreater{} \sphinxbfcode{mJobs}}
Lista de \sphinxtitleref{Job}, que contienen la información acerca de las experiencias laborales.

\end{fulllineitems}



\subsection{Constructor}
\label{Adapter/JobsAdapter:constructor}

\subsubsection{JobsAdapter}
\label{Adapter/JobsAdapter:id1}\index{JobsAdapter(List) (Java constructor)}

\begin{fulllineitems}
\phantomsection\label{Adapter/JobsAdapter:com.fiuba.tallerii.jobify.JobsAdapter.JobsAdapter(List)}\pysiglinewithargsret{public \sphinxbfcode{JobsAdapter}}{List\textless{}{\hyperref[Model/Job:com.fiuba.tallerii.jobify.Job]{\sphinxcrossref{Job}}}\textgreater{}\emph{ jobs}}{}
Inicializa la lista de \sphinxtitleref{Job} a manipular.
\begin{quote}\begin{description}
\item[{Parameters}] \leavevmode\begin{itemize}
\item {} 
\textbf{\texttt{jobs}} -- 

\end{itemize}

\end{description}\end{quote}

\end{fulllineitems}



\subsection{Methods}
\label{Adapter/JobsAdapter:methods}

\subsubsection{onCreateViewHolder}
\label{Adapter/JobsAdapter:oncreateviewholder}\index{onCreateViewHolder(ViewGroup, int) (Java method)}

\begin{fulllineitems}
\phantomsection\label{Adapter/JobsAdapter:com.fiuba.tallerii.jobify.JobsAdapter.onCreateViewHolder(ViewGroup, int)}\pysiglinewithargsret{public JobsViewHolder \sphinxbfcode{onCreateViewHolder}}{ViewGroup\emph{ parent}, int\emph{ viewType}}{}
Crea el View para mostrar un Job de la forma deseada y crea con el mismo un {\color{red}\bfseries{}{}`}JobsViewHolder'
\begin{quote}\begin{description}
\item[{Parameters}] \leavevmode\begin{itemize}
\item {} 
\textbf{\texttt{parent}} -- 

\item {} 
\textbf{\texttt{viewType}} -- 

\end{itemize}

\end{description}\end{quote}

\end{fulllineitems}



\subsubsection{getItemCount}
\label{Adapter/JobsAdapter:getitemcount}\index{getItemCount() (Java method)}

\begin{fulllineitems}
\phantomsection\label{Adapter/JobsAdapter:com.fiuba.tallerii.jobify.JobsAdapter.getItemCount()}\pysiglinewithargsret{public int \sphinxbfcode{getItemCount}}{}{}
Devuelve la cantidad de Job's que se están manipulando

\end{fulllineitems}



\subsubsection{onBindViewHolder}
\label{Adapter/JobsAdapter:onbindviewholder}\index{onBindViewHolder(JobsViewHolder, int) (Java method)}

\begin{fulllineitems}
\phantomsection\label{Adapter/JobsAdapter:com.fiuba.tallerii.jobify.JobsAdapter.onBindViewHolder(JobsViewHolder, int)}\pysiglinewithargsret{public void \sphinxbfcode{onBindViewHolder}}{JobsViewHolder\emph{ holder}, int\emph{ position}}{}
Rellena los valores del JobsViewHolder para que pueda ser mostrado de la forma deseada.
\begin{quote}\begin{description}
\item[{Parameters}] \leavevmode\begin{itemize}
\item {} 
\textbf{\texttt{holder}} -- JobsViewHolder que se está agregando

\item {} 
\textbf{\texttt{position}} -- índice en la lista de Jobs, del \sphinxtitleref{Job} que está siendo agregado

\end{itemize}

\end{description}\end{quote}

\end{fulllineitems}



\section{ContactsAdapter}
\label{Adapter/ContactsAdapter:contactsadapter}\label{Adapter/ContactsAdapter::doc}\index{ContactsAdapter (Java class)}

\begin{fulllineitems}
\phantomsection\label{Adapter/ContactsAdapter:com.fiuba.tallerii.jobify.ContactsAdapter}\pysigline{private class \sphinxbfcode{ContactsAdapter} extends RecyclerView.Adapter\textless{}ContactViewHolder\textgreater{}}
Adaptador definido internamente por \sphinxtitleref{ContactsFragment} para controlar la vista de los contactos en forma de un \sphinxtitleref{ContactsViewHolder}

\end{fulllineitems}



\subsection{Fields}
\label{Adapter/ContactsAdapter:fields}

\subsubsection{mContacts}
\label{Adapter/ContactsAdapter:mcontacts}\index{mContacts (Java field)}

\begin{fulllineitems}
\phantomsection\label{Adapter/ContactsAdapter:com.fiuba.tallerii.jobify.ContactsAdapter.mContacts}\pysigline{private List\textless{}{\hyperref[Model/Contact:com.fiuba.tallerii.jobify.Contact]{\sphinxcrossref{Contact}}}\textgreater{} \sphinxbfcode{mContacts}}
Lista de \sphinxtitleref{Contact}, que contienen la información acerca de los contactos del usuario.

\end{fulllineitems}



\subsection{Constructor}
\label{Adapter/ContactsAdapter:constructor}

\subsubsection{ContactsAdapter}
\label{Adapter/ContactsAdapter:id1}\index{ContactsAdapter(List) (Java constructor)}

\begin{fulllineitems}
\phantomsection\label{Adapter/ContactsAdapter:com.fiuba.tallerii.jobify.ContactsAdapter.ContactsAdapter(List)}\pysiglinewithargsret{public \sphinxbfcode{ContactsAdapter}}{List\textless{}{\hyperref[Model/Contact:com.fiuba.tallerii.jobify.Contact]{\sphinxcrossref{Contact}}}\textgreater{}\emph{ contacts}}{}
Inicializa la lista de \sphinxtitleref{Contact} a manipular.
\begin{quote}\begin{description}
\item[{Parameters}] \leavevmode\begin{itemize}
\item {} 
\textbf{\texttt{contacts}} -- 

\end{itemize}

\end{description}\end{quote}

\end{fulllineitems}



\subsection{Methods}
\label{Adapter/ContactsAdapter:methods}

\subsubsection{onCreateViewHolder}
\label{Adapter/ContactsAdapter:oncreateviewholder}\index{onCreateViewHolder(ViewGroup, int) (Java method)}

\begin{fulllineitems}
\phantomsection\label{Adapter/ContactsAdapter:com.fiuba.tallerii.jobify.ContactsAdapter.onCreateViewHolder(ViewGroup, int)}\pysiglinewithargsret{public ContactsViewHolder \sphinxbfcode{onCreateViewHolder}}{ViewGroup\emph{ parent}, int\emph{ viewType}}{}
Crea el View para mostrar un Contact de la forma deseada y crea con el mismo un {\color{red}\bfseries{}{}`}ContactsViewHolder'
\begin{quote}\begin{description}
\item[{Parameters}] \leavevmode\begin{itemize}
\item {} 
\textbf{\texttt{parent}} -- 

\item {} 
\textbf{\texttt{viewType}} -- 

\end{itemize}

\end{description}\end{quote}

\end{fulllineitems}



\subsubsection{getItemCount}
\label{Adapter/ContactsAdapter:getitemcount}\index{getItemCount() (Java method)}

\begin{fulllineitems}
\phantomsection\label{Adapter/ContactsAdapter:com.fiuba.tallerii.jobify.ContactsAdapter.getItemCount()}\pysiglinewithargsret{public int \sphinxbfcode{getItemCount}}{}{}
Devuelve la cantidad de Contact's que se están manipulando

\end{fulllineitems}



\subsubsection{onBindViewHolder}
\label{Adapter/ContactsAdapter:onbindviewholder}\index{onBindViewHolder(ContactsViewHolder, int) (Java method)}

\begin{fulllineitems}
\phantomsection\label{Adapter/ContactsAdapter:com.fiuba.tallerii.jobify.ContactsAdapter.onBindViewHolder(ContactsViewHolder, int)}\pysiglinewithargsret{public void \sphinxbfcode{onBindViewHolder}}{ContactsViewHolder\emph{ holder}, int\emph{ position}}{}
Rellena los valores del ContactsViewHolder para que pueda ser mostrado de la forma deseada.
\begin{quote}\begin{description}
\item[{Parameters}] \leavevmode\begin{itemize}
\item {} 
\textbf{\texttt{holder}} -- ContactsViewHolder que se está agregando

\item {} 
\textbf{\texttt{position}} -- índice en la lista de Contacts, del \sphinxtitleref{Contact} que está siendo agregado

\end{itemize}

\end{description}\end{quote}

\end{fulllineitems}



\section{NotificationAdapter}
\label{Adapter/NotificationsAdapter:notificationadapter}\label{Adapter/NotificationsAdapter::doc}

\subsection{Fields}
\label{Adapter/NotificationsAdapter:fields}

\subsubsection{mNotifications}
\label{Adapter/NotificationsAdapter:mnotifications}\index{mNotifications (Java field)}

\begin{fulllineitems}
\phantomsection\label{Adapter/NotificationsAdapter:com.fiuba.tallerii.jobify.NotificationAdapter.mNotifications}\pysigline{private List\textless{}{\hyperref[Model/Notification:com.fiuba.tallerii.jobify.Notification]{\sphinxcrossref{Notification}}}\textgreater{} \sphinxbfcode{mNotifications}}
Lista de \sphinxtitleref{Notification}, que contienen la información acerca de las notificaciones del usuario.

\end{fulllineitems}



\subsection{Constructor}
\label{Adapter/NotificationsAdapter:constructor}

\subsubsection{NotificationAdapter}
\label{Adapter/NotificationsAdapter:id1}\index{NotificationAdapter(List) (Java constructor)}

\begin{fulllineitems}
\phantomsection\label{Adapter/NotificationsAdapter:com.fiuba.tallerii.jobify.NotificationAdapter.NotificationAdapter(List)}\pysiglinewithargsret{public \sphinxbfcode{NotificationAdapter}}{List\textless{}{\hyperref[Model/Notification:com.fiuba.tallerii.jobify.Notification]{\sphinxcrossref{Notification}}}\textgreater{}\emph{ notifications}}{}
Inicializa la lista de \sphinxtitleref{Notification} a manipular.
\begin{quote}\begin{description}
\item[{Parameters}] \leavevmode\begin{itemize}
\item {} 
\textbf{\texttt{notifications}} -- 

\end{itemize}

\end{description}\end{quote}

\end{fulllineitems}



\subsection{Methods}
\label{Adapter/NotificationsAdapter:methods}

\subsubsection{onCreateViewHolder}
\label{Adapter/NotificationsAdapter:oncreateviewholder}\index{onCreateViewHolder(ViewGroup, int) (Java method)}

\begin{fulllineitems}
\phantomsection\label{Adapter/NotificationsAdapter:com.fiuba.tallerii.jobify.NotificationAdapter.onCreateViewHolder(ViewGroup, int)}\pysiglinewithargsret{public NotificationsViewHolder \sphinxbfcode{onCreateViewHolder}}{ViewGroup\emph{ parent}, int\emph{ viewType}}{}
Crea el View para mostrar un \sphinxtitleref{Notification} de la forma deseada y crea con el mismo, un {\color{red}\bfseries{}{}`}NotificationsViewHolder'
\begin{quote}\begin{description}
\item[{Parameters}] \leavevmode\begin{itemize}
\item {} 
\textbf{\texttt{parent}} -- 

\item {} 
\textbf{\texttt{viewType}} -- 

\end{itemize}

\end{description}\end{quote}

\end{fulllineitems}



\subsubsection{getItemCount}
\label{Adapter/NotificationsAdapter:getitemcount}\index{getItemCount() (Java method)}

\begin{fulllineitems}
\phantomsection\label{Adapter/NotificationsAdapter:com.fiuba.tallerii.jobify.NotificationAdapter.getItemCount()}\pysiglinewithargsret{public int \sphinxbfcode{getItemCount}}{}{}
Devuelve la cantidad de Notification's que se están manipulando

\end{fulllineitems}



\subsubsection{onBindViewHolder}
\label{Adapter/NotificationsAdapter:onbindviewholder}\index{onBindViewHolder(NotificationsViewHolder, int) (Java method)}

\begin{fulllineitems}
\phantomsection\label{Adapter/NotificationsAdapter:com.fiuba.tallerii.jobify.NotificationAdapter.onBindViewHolder(NotificationsViewHolder, int)}\pysiglinewithargsret{public void \sphinxbfcode{onBindViewHolder}}{NotificationsViewHolder\emph{ holder}, int\emph{ position}}{}
Rellena los valores del NotificationsViewHolder para que pueda ser mostrado de la forma deseada.
\begin{quote}\begin{description}
\item[{Parameters}] \leavevmode\begin{itemize}
\item {} 
\textbf{\texttt{holder}} -- NotificationsViewHolder que se está agregando

\item {} 
\textbf{\texttt{position}} -- índice en la lista de Notifications, del \sphinxtitleref{Notification} que está siendo agregado.

\end{itemize}

\end{description}\end{quote}

\end{fulllineitems}



\chapter{Model}
\label{Model/package-index:model}\label{Model/package-index::doc}

\section{User}
\label{Model/User::doc}\label{Model/User:user}\index{User (Java class)}

\begin{fulllineitems}
\phantomsection\label{Model/User:com.fiuba.tallerii.jobify.User}\pysigline{public class \sphinxbfcode{User}}
Clase que contiene información del usuario

\end{fulllineitems}



\subsection{Constructor}
\label{Model/User:constructor}

\subsubsection{User}
\label{Model/User:id1}\index{User(String, String, String, String) (Java constructor)}

\begin{fulllineitems}
\phantomsection\label{Model/User:com.fiuba.tallerii.jobify.User.User(String, String, String, String)}\pysiglinewithargsret{public \sphinxbfcode{User}}{\href{http://docs.oracle.com/javase/6/docs/api/java/lang/String.html}{String}\emph{ firstName}, \href{http://docs.oracle.com/javase/6/docs/api/java/lang/String.html}{String}\emph{ lastName}, \href{http://docs.oracle.com/javase/6/docs/api/java/lang/String.html}{String}\emph{ email}, \href{http://docs.oracle.com/javase/6/docs/api/java/lang/String.html}{String}\emph{ password}}{}
Constructor. Inicializa los parametros ingresados.
\begin{quote}\begin{description}
\item[{Parameters}] \leavevmode\begin{itemize}
\item {} 
\textbf{\texttt{firstName}} -- primer nombre del usuario.

\item {} 
\textbf{\texttt{lastName}} -- apellido del usuario.

\item {} 
\textbf{\texttt{email}} -- email del usuario.

\item {} 
\textbf{\texttt{password}} -- password del usuario.

\end{itemize}

\end{description}\end{quote}

\end{fulllineitems}



\subsection{Methods}
\label{Model/User:methods}
Getters y Setters de los atributos.


\section{Contact}
\label{Model/Contact:contact}\label{Model/Contact::doc}\index{Contact (Java class)}

\begin{fulllineitems}
\phantomsection\label{Model/Contact:com.fiuba.tallerii.jobify.Contact}\pysigline{public class \sphinxbfcode{Contact}}
Clase que contiene información acerca de un contacto.

\end{fulllineitems}



\subsection{Contact}
\label{Model/Contact:id1}

\subsubsection{Notification}
\label{Model/Contact:notification}\index{Contact(String, String, Bitmap) (Java constructor)}

\begin{fulllineitems}
\phantomsection\label{Model/Contact:com.fiuba.tallerii.jobify.Contact.Contact(String, String, Bitmap)}\pysiglinewithargsret{public \sphinxbfcode{Contact}}{\href{http://docs.oracle.com/javase/6/docs/api/java/lang/String.html}{String}\emph{ name}, \href{http://docs.oracle.com/javase/6/docs/api/java/lang/String.html}{String}\emph{ email}, Bitmap\emph{ pictureBitmap}}{}
Constructor. Inicializa los parametros ingresados.
\begin{quote}\begin{description}
\item[{Parameters}] \leavevmode\begin{itemize}
\item {} 
\textbf{\texttt{name}} -- nombre completo del contacto. Incluye apellido

\item {} 
\textbf{\texttt{email}} -- email del contacto

\item {} 
\textbf{\texttt{pictureBitmap}} -- foto de perfil del contacto

\end{itemize}

\end{description}\end{quote}

\end{fulllineitems}



\subsection{Methods}
\label{Model/Contact:methods}
Getters y Setters de los atributos.


\subsubsection{addSkill}
\label{Model/Contact:addskill}\index{addSkill(Skill) (Java method)}

\begin{fulllineitems}
\phantomsection\label{Model/Contact:com.fiuba.tallerii.jobify.Contact.addSkill(Skill)}\pysiglinewithargsret{public void \sphinxbfcode{addSkill}}{{\hyperref[Model/Skill:com.fiuba.tallerii.jobify.Skill]{\sphinxcrossref{Skill}}}\emph{ skillToAdd}}{}
Agrega una destreza ignorando duplicados.
\begin{quote}\begin{description}
\item[{Parameters}] \leavevmode\begin{itemize}
\item {} 
\textbf{\texttt{skillToAdd}} -- destreza a añadir.

\end{itemize}

\end{description}\end{quote}

\end{fulllineitems}



\subsubsection{addJob}
\label{Model/Contact:addjob}\index{addJob(Job) (Java method)}

\begin{fulllineitems}
\phantomsection\label{Model/Contact:com.fiuba.tallerii.jobify.Contact.addJob(Job)}\pysiglinewithargsret{public void \sphinxbfcode{addJob}}{{\hyperref[Model/Job:com.fiuba.tallerii.jobify.Job]{\sphinxcrossref{Job}}}\emph{ jobToAdd}}{}
Agrega una experiencia laboral ignorando duplicados.
\begin{quote}\begin{description}
\item[{Parameters}] \leavevmode\begin{itemize}
\item {} 
\textbf{\texttt{jobToAdd}} -- experiencia laboral a añadir.

\end{itemize}

\end{description}\end{quote}

\end{fulllineitems}



\section{Skill}
\label{Model/Skill:skill}\label{Model/Skill::doc}\index{Skill (Java class)}

\begin{fulllineitems}
\phantomsection\label{Model/Skill:com.fiuba.tallerii.jobify.Skill}\pysigline{public class \sphinxbfcode{Skill}}
Clase que representa una destreza del usuario

\end{fulllineitems}



\subsection{Constructor}
\label{Model/Skill:constructor}

\subsubsection{Skill}
\label{Model/Skill:id1}\index{Skill(String, String, String) (Java constructor)}

\begin{fulllineitems}
\phantomsection\label{Model/Skill:com.fiuba.tallerii.jobify.Skill.Skill(String, String, String)}\pysiglinewithargsret{public \sphinxbfcode{Skill}}{\href{http://docs.oracle.com/javase/6/docs/api/java/lang/String.html}{String}\emph{ tittle}, \href{http://docs.oracle.com/javase/6/docs/api/java/lang/String.html}{String}\emph{ category}, \href{http://docs.oracle.com/javase/6/docs/api/java/lang/String.html}{String}\emph{ description}}{}
Constructor. Inicializa los parametros ingresados.
\begin{quote}\begin{description}
\item[{Parameters}] \leavevmode\begin{itemize}
\item {} 
\textbf{\texttt{tittle}} -- título de la destreza.

\item {} 
\textbf{\texttt{category}} -- categoría de la destreza.

\item {} 
\textbf{\texttt{description}} -- datos adicionales de la destreza.

\end{itemize}

\end{description}\end{quote}

\end{fulllineitems}



\subsection{Methods}
\label{Model/Skill:methods}
Getters y Setters de los atributos.


\section{Job}
\label{Model/Job:job}\label{Model/Job::doc}\index{Job (Java class)}

\begin{fulllineitems}
\phantomsection\label{Model/Job:com.fiuba.tallerii.jobify.Job}\pysigline{public class \sphinxbfcode{Job}}
Clase que representa una experiencia laboral del usuario.

\end{fulllineitems}



\subsection{Constructor}
\label{Model/Job:constructor}

\subsubsection{Job}
\label{Model/Job:id1}\index{Skill(String, String, String) (Java constructor)}

\begin{fulllineitems}
\phantomsection\label{Model/Job:com.fiuba.tallerii.jobify.Job.Skill(String, String, String)}\pysiglinewithargsret{public \sphinxbfcode{Skill}}{\href{http://docs.oracle.com/javase/6/docs/api/java/lang/String.html}{String}\emph{ tittle}, \href{http://docs.oracle.com/javase/6/docs/api/java/lang/String.html}{String}\emph{ category}, \href{http://docs.oracle.com/javase/6/docs/api/java/lang/String.html}{String}\emph{ description}}{}
Constructor. Inicializa los parametros ingresados.
\begin{quote}\begin{description}
\item[{Parameters}] \leavevmode\begin{itemize}
\item {} 
\textbf{\texttt{tittle}} -- título de la experiencia laboral.

\item {} 
\textbf{\texttt{category}} -- categoría de la experiencia laboral.

\item {} 
\textbf{\texttt{description}} -- datos adicionales de la experiencia laboral.

\end{itemize}

\end{description}\end{quote}

\end{fulllineitems}



\subsection{Methods}
\label{Model/Job:methods}
Getters y Setters de los atributos.


\section{Notification}
\label{Model/Notification:notification}\label{Model/Notification::doc}\index{Notification (Java class)}

\begin{fulllineitems}
\phantomsection\label{Model/Notification:com.fiuba.tallerii.jobify.Notification}\pysigline{public class \sphinxbfcode{Notification}}
Clase que contiene información acerca de una notificación.

\end{fulllineitems}



\subsection{Constructor}
\label{Model/Notification:constructor}

\subsubsection{Notification}
\label{Model/Notification:id1}\index{Notification(String, int) (Java constructor)}

\begin{fulllineitems}
\phantomsection\label{Model/Notification:com.fiuba.tallerii.jobify.Notification.Notification(String, int)}\pysiglinewithargsret{public \sphinxbfcode{Notification}}{\href{http://docs.oracle.com/javase/6/docs/api/java/lang/String.html}{String}\emph{ content}, int\emph{ code}}{}
Constructor. Inicializa los parametros ingresados.
\begin{quote}\begin{description}
\item[{Parameters}] \leavevmode\begin{itemize}
\item {} 
\textbf{\texttt{content}} -- contenido de la notificación.

\item {} 
\textbf{\texttt{code}} -- código interno que identifica de que tipo de notificación se trata.

\end{itemize}

\end{description}\end{quote}

\end{fulllineitems}



\subsection{Methods}
\label{Model/Notification:methods}
Getters y Setters de los atributos.


\subsubsection{generateTitle}
\label{Model/Notification:generatetitle}\index{generateTitle(int) (Java method)}

\begin{fulllineitems}
\phantomsection\label{Model/Notification:com.fiuba.tallerii.jobify.Notification.generateTitle(int)}\pysiglinewithargsret{private \href{http://docs.oracle.com/javase/6/docs/api/java/lang/String.html}{String} \sphinxbfcode{generateTitle}}{int\emph{ code}}{}
Genera el título dependiendo dl código de la notificación.
\begin{quote}\begin{description}
\item[{Parameters}] \leavevmode\begin{itemize}
\item {} 
\textbf{\texttt{code}} -- código que identifica el tipo de notificación

\end{itemize}

\end{description}\end{quote}

\end{fulllineitems}



\chapter{Utils}
\label{Utils/package-index:utils}\label{Utils/package-index::doc}

\section{FieldValidator}
\label{Utils/FieldValidator:fieldvalidator}\label{Utils/FieldValidator::doc}\index{FieldValidator (Java class)}

\begin{fulllineitems}
\phantomsection\label{Utils/FieldValidator:com.fiuba.tallerii.jobify.FieldValidator}\pysigline{public class \sphinxbfcode{FieldValidator}}
Clase que permite verificar si un campo es válido o no

\end{fulllineitems}



\subsection{Methods}
\label{Utils/FieldValidator:methods}

\subsubsection{isNameValid}
\label{Utils/FieldValidator:isnamevalid}\index{isNameValid(String) (Java method)}

\begin{fulllineitems}
\phantomsection\label{Utils/FieldValidator:com.fiuba.tallerii.jobify.FieldValidator.isNameValid(String)}\pysiglinewithargsret{public boolean \sphinxbfcode{isNameValid}}{\href{http://docs.oracle.com/javase/6/docs/api/java/lang/String.html}{String}\emph{ name}}{}
Devuelve true si la cadena ingresada contiene solo letras del alfabeto. Devuelve false en otro caso.
\begin{quote}\begin{description}
\item[{Parameters}] \leavevmode\begin{itemize}
\item {} 
\textbf{\texttt{name}} -- cadena a verificar

\end{itemize}

\end{description}\end{quote}

\end{fulllineitems}



\subsubsection{isEmailValid}
\label{Utils/FieldValidator:isemailvalid}\index{isEmailValid(String) (Java method)}

\begin{fulllineitems}
\phantomsection\label{Utils/FieldValidator:com.fiuba.tallerii.jobify.FieldValidator.isEmailValid(String)}\pysiglinewithargsret{public boolean \sphinxbfcode{isEmailValid}}{\href{http://docs.oracle.com/javase/6/docs/api/java/lang/String.html}{String}\emph{ email}}{}
Devuelve true si la cadena ingresada posee el formato de una dirección mail. Devuelve false en otro caso.
\begin{quote}\begin{description}
\item[{Parameters}] \leavevmode\begin{itemize}
\item {} 
\textbf{\texttt{email}} -- cadena a verificar

\end{itemize}

\end{description}\end{quote}

\end{fulllineitems}



\subsubsection{isPasswordValid}
\label{Utils/FieldValidator:ispasswordvalid}\index{isPasswordValid(String) (Java method)}

\begin{fulllineitems}
\phantomsection\label{Utils/FieldValidator:com.fiuba.tallerii.jobify.FieldValidator.isPasswordValid(String)}\pysiglinewithargsret{public boolean \sphinxbfcode{isPasswordValid}}{\href{http://docs.oracle.com/javase/6/docs/api/java/lang/String.html}{String}\emph{ password}}{}
Devuelve true si la cadena ingresada posee 4 o más caracteres. Devuelve false en otro caso.
\begin{quote}\begin{description}
\item[{Parameters}] \leavevmode\begin{itemize}
\item {} 
\textbf{\texttt{password}} -- cadena a verificar

\end{itemize}

\end{description}\end{quote}

\end{fulllineitems}



\chapter{Indices and tables}
\label{index:indices-and-tables}\begin{itemize}
\item {} 
\DUrole{xref,std,std-ref}{genindex}

\item {} 
\DUrole{xref,std,std-ref}{modindex}

\item {} 
\DUrole{xref,std,std-ref}{search}

\end{itemize}



\renewcommand{\indexname}{Index}
\printindex
\end{document}
